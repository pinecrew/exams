\question{Решение задачи конвективного переноса методом преобразования Лапласа.}

Рассмотрим задачу о конвективном переносе:
\[
    \left\{ \begin{array}{l}
        \ds \pder{u}{t} = -v\pder{u}{x},\quad x \in (0;\ +\infty), \\
        u(0, t) = p, \\
        u(x, 0) = 0.
    \end{array} \right.
\]

Сделаем преобразование Лапласа по времени: \( u\to\bar{u} \),
\( \ds\pder{u}{t} \to s\bar{u} \).

Преобразованная задача:
\[
    \left\{ \begin{array}{l}
        \ds s\bar{u} = -v\pder{\bar{u}}{x}, \\
        \bar{u}(0) = p/s.
    \end{array} \right.
\]

Её решение: \( \ds \bar{u}(x) = \frac{p}{s}\cdot\e^{-\frac{sx}{v}} \).

Сделаем обратное преобразование:
\[
    u(x, t) = p\cdot\eta\left(t - \frac{x}{v}\right),
    \text{ где \( \eta(x) \) -- функция Хевисайда}.
\]

Таким образом, решение может быть записано в виде:
\[
    u(x, t) = \left\{ \begin{array}{l}
        0,\text{ если } t < x/v, \\ p,\text{ если } t \ge x/v.
    \end{array} \right.
\]

Наглядно себе это можно представить как распространение возмущения \( p \)
вдоль оси \( Ox \) со скоростью \( v \).
\newpage
