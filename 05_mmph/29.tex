\question{Нелинейные ДУЧП I. Вывод и применение закона сохранения к задаче о
дорожном движении.}

Получим закон сохранения в одномерном случае. Рассмотрим перенос свойства
\( u \) вдоль оси \( Ox \).\\
Изменение \( u \) на отрезке \( [a, b] \) в единицу времени есть ни что иное,
как \( \ds \der{}{t} \int\limits_a^b u(x, t)\d x \).\\
С другой стороны это изменение связано с потоком \( f \) свойства \( u \) через
границы отрезка:
\[
   f(a, t) - f(b, t) = -\int\limits_a^b \pder{f}{x}\d x.
\]

Приравнивая эти два выражения, имеем \( \ds \pder{u}{t} + \pder{f}{x} = 0 \),
являющееся математической записью закона сохранения свойства \( u \) в локальной форме.

\subquestion{Применение закона сохранения к задаче о дорожном движении}

Пусть зависимость потока от плотности автомобилей квадратичная, то есть
\( f(u) = u^2 \). Тогда по правилу дифференцирования сложной функции:
\( \ds \pder{f}{x} = \pder{f}{u}\pder{u}{x} = 2u\pder{u}{x} \), а закон
сохранения можно записать в виде
\[
    \pder{u}{t} + 2u\pder{u}{x} = 0.
\]

Учитывая начальную плотность распределения автомобилей \( \ds u(x, 0) = 
\left\{ \begin{array}{rl}
    1, & \text{ при } x \le 0, \\[-.6em]
    1-x, & \text{ при } x \in (0, 1), \\[-.6em]
    0, & \text{ при } x \ge 1;
\end{array} \right. \)
получаем задачу Коши:
\[
    \left\{ \begin{array}{l}
        \ds \pder{u}{t} + 2u\pder{u}{x} = 0, \\[.4em]
        u(x, 0) = \left\{ \begin{array}{rl}
            1, & \text{ при } x \le 0, \\[-.6em]
            1-x, & \text{ при } x \in (0, 1), \\[-.6em]
            0, & \text{ при } x \ge 1;
        \end{array} \right.
    \end{array} \right.
\]

Как известно, \( u(x, t) \) не меняется вдоль характеристики. Если известна
начальная концентрация \( u(x_0, 0) \), то уравнение характеристик, начинающейся
в точке \( (x_0, 0) \) принимает вид: \( x = x_0 + g(u(x_0, 0))t \).

Для \( x_0 \le 0 \) получаем:
\( x = x_0 + g\bigl(u(x_0, 0)\bigr)t = x_0 + g\bigl(1\bigr)t = x_0 + 2t \).\\
Решая относительно \( t \), получаем \( \ds t = \frac{1}{2}(x - x_0) \).

Для \( x_0 \in (0, 1) \) получаем
\( x = x_0 + g\bigl(1-x_0\bigr)t = x_0 + 2(1-x_0)t \).\\
Решая относительно \( t \), получаем \( \ds t = \frac{x-x_0}{2(1 - x_0)} \).

Для \( x_0 \ge 1 \) получаем \( x = x_0 + g\bigl(0\bigr)t = x_0 \) --
вертикальные линии, выходящие из точек \( x_0 \). Все характеристики изображены
на рис. \smiley.

Отметим, что характеристики сходятся в одну точку при \( t = 1/2 \). Значит, для
решения задачи при \( t > 1/2 \) необходимо применять другой метод. Когда
характеристики сходятся в одной точке, говорят об ударной волне (разрыве
решения). Передний фронт ударной волны будет перемещаться вдоль автострады со
следующей скоростью:
\[
    S = \frac{f(u_R) - f(u_L)}{u_R - u_L},
\]
где \( u_R \) и \( u_L \) -- значения решения справа и слева от волнового
фронта, а \( f(u_R) \) и \( f(u_L) \) -- значения потока при этих плотностях. В
поставленной задаче Коши \( S = (0 - 1)/(0 - 1) = 1 \).

\newpage
