\question{Классификация сил, действующих на механическую систему: силы внутренние
и внешние, задаваемые силы. Главный вектор и главный момент внутренних сил.
Дифференциальные уравнения движения механической системы.}

Силы, действующие на механическую систему можно разделить на две категории:
внешние и внутренние. Внешними силами называют силы, действующие на систему со
стороны других систем, а внутренними -- силы, действующие на части системы со
стороны других частей этой же системы.

Свойства внутренних сил:
\begin{enumerate}
    \item главный вектор внутренних сил равен 0: \( \vec{F}^i = 0 \), так
    как по второму закону Ньютона для \( i \)-й и \( j \)-й точек системы имеем:
    \[
        \vec{F}^i_{ij} = -\vec{F}^i_{ji}, \quad \text{следовательно,}\quad
        \sum_i\sum_k \vec{F}^i_{ij} = 0, \quad \text{или:} \quad
        \vec{F}^i = 0.
    \]
    
    \item главный момент внутренних сил равен 0: \( \vec{M}^i = 0 \). Это
    следует из тех же соображений, так как \( \vec{F}^i_{ij} \) и
    \( \vec{F}^i_{ji} \) лежат на одной прямой.
\end{enumerate}

Дифференциальные уравнения движения механической системы зависят от вида
движения. Для общности рассмотрим свободное движение системы. Его можно
рассмотреть как суперпозицию поступательного движения тела с центром масс и
вращательного вокруг центра масс. Поэтому дифференциальные уравнения движения
имеют вид:
\begin{gather*}
    M\vec{a}_c = \vec{F}^e,\\
    \der{}{t}(\hat{I}\cdot\vec{\omega}) = \vec{M}^e_c,
\end{gather*}
где \( \hat{I} \) -- тензор инерции механической системы.
\newpage
