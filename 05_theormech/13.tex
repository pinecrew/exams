\chapter{Принцип наименьшего действия Мопертюи-Лагранжа. Действие по Лагранжу.}

\header{Принцип наименьшего действия Мопертюи}:

\emph{для действительного движения частицы интеграл
\( \ds \int\limits_a^b v\d S \), взятый по прямой, есть минимум по сравнению
с такими же интегралами по другим кривым.}

Запишем действие по Гамильтону для системы с постоянной полной энергией
\( E = T + \varPi = h = \const \) и идеальными склерономными связями:
\[
    S = \int\limits_0^t L\d t = \int\limits_0^t (T - \varPi)\d t =
    \int\limits_0^t (2T - h)\d t = \int\limits_0^t 2T\d t - ht.
\]

Интеграл \( W = \int\limits_0^t 2T\d t \) называется действием по Лагранжу.
Тогда действие по Гамильтону: \( S = W - ht \).

\header{Принцип наименьшего действия Мопертюи-Лагранжа}:

\emph{действительные движения голономной, консервативной механической системы
между двумя заданными конфигурациями \( A(q_0, q_1, \ldots) \) и \( B(q_0, q_1,
\ldots) \) отличаются от кинематически возможных движений, совершаемых между
теми же конфигурациями и с той же полной энергией, тем, что для действительного
движения полная вариация действия по Лагранжу равна нулю: \( \D W = 0 \).}

Покажем, что из этого принципа следуют уравнения Лагранжа второго рода.

Распишем полную вариацию действия по Лагранжу:
\[
    \D W = \D S + h\D t = \delta S + \dot{S}\D t + h\D t =
    \delta S + (L + h)\D t = \delta S + 2T\D t.
\]
Учтем, что \( \ds
    \delta S = \int\limits_0^t \left[\sum\limits_{i=1}^n \left\{\pder{L}{q_i} -
    \der{}{t}\left(\pder{L}{\dot{q}_i}\right)\right\}\delta q_i\right]\d t +
    \sum\limits_{i=1}^n \pder{L}{\dot{q}_i}\delta q_i\Biggr|_0^t \),
где
\[
    \sum\limits_{i=1}^n \pder{L}{\dot{q}_i}\delta q_i\biggr|_0^t =
    \sum\limits_{i=1}^n \pder{T}{\dot{q}_i}\delta q_i\biggr|_0^t =
    \sum\limits_{i=1}^n \pder{T}{\dot{q}_i}\D q_i\biggr|_0^t -
    \sum\limits_{i=1}^n \pder{T}{\dot{q}_i}\dot{q}_i\D t\biggr|_0^t =
    -2T\D t\biggr|_0^t.
\]
Здесь учтено, что \( \D q_n = 0 \) в начале и конце движения. Подставляя
обратно:
\[
    \D W = \int\limits_0^t \left[\sum\limits_{i=1}^n \left\{\pder{L}{q_i} -
    \der{}{t}\left(\pder{L}{\dot{q}_i}\right)\right\}\delta q_i\right]\d t +
    2T\D t - 2T\D t = \int\limits_0^t \left[\sum\limits_{i=1}^n \left\{
    \pder{L}{q_i} - \der{}{t}\left(\pder{L}{\dot{q}_i}\right)\right\}
    \delta q_i\right]\d t = 0.
\]

Отсюда очевидным образом следуют уравнения Лагранжа второго рода:
\[
    \sum\limits_{i=1}^n \left[\pder{L}{q_i} -
    \der{}{t}\left(\pder{L}{\dot{q}_i}\right)\right] = 0.
\]

\newpage
