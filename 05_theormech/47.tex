\question{Теорема о движении центра масс механической системы. Следствия из
теоремы. Дифференциальные уравнения поступательного движения твердого
тела.}

\subquestion{Теорема о движении центра масс механической системы}
Центр масс механической системы движется как точка, масса которой равна массе
всей системы \( M = \sum m_k \), к которой приложены все внешние силы системы:
\( M\vec{a}_c = \sum \vec{F}^e_k \). В координатной форме:
\( M\ddot{x}_c = \sum F^e_{kx} \), \( M\ddot{y}_c = \sum F^e_{ky} \),
\( M\ddot{z}_c = \sum F^e_{kz} \), где \( \vec{a}_c \), \( \ddot{x}_c \),
\( \ddot{y}_c \), \( \ddot{z}_c \) -- ускорение центра масс и его проекции на
декартовы оси, \( \vec{F}^e_k \), \( F^e_{kx} \), \( F^e_{ky} \), \( F^e_{kz} \)
-- внешняя сила и её проекции на декартовы оси.

\subquestion{Следствия}
\begin{enumerate}
    \item Если главный вектор внешних сил, приложенных к механической системе,
    равен нулю, то центр масс системы находится в покое или движется равномерно
    и прямолинейно, так как ускорение центра масс равно нулю:
    \( \vec{a}_c = 0 \).
    
    \item Если проекция главного вектора внешних сил на какую-нибудь ось равна
    нулю, то центр масс системы или не изменяет своего положения относительно
    данной оси, или движется относительно неё равномерно.
\end{enumerate}

\subquestion{Дифференциальные уравнения поступательного движения твердого тела}

Пусть тело под действием приложенных к нему сил движется поступательно. Применяя
теорему о движении центра масс (центра инерции), можно получить дифференциальные
уравнения поступательного движения тела: \( m\vec{a}_c = \sum \vec{F}^e_k \), где
\( m \) -- масса тела, \( \vec{a}_c \) -- ускорение его центра инерции,
\( \sum \vec{F}^e_k \) -- главный вектор внешних сил, приложенных к телу.

В проекциях на оси координат получим \( m\ddot{x}_c = \sum F^e_{kx} \),
\( m\ddot{y}_c = \sum F^e_{ky} \), \( m\ddot{z}_c = \sum F^e_{kz} \).

Интегрируя эти уравнения, можно определить координаты центра инерции тела как
функции времени. Постоянные интегрирования определяются из начальных условий
движения (при \( t = t_0 \): \( x_c = x_{C_0},\ y_c = y_{C_0},\ z_c = z_{C_0},\ 
\dot{x}_c = \dot{x}_{C_0},\ \dot{y}_c = \dot{y}_{C_0},\ 
\dot{z}_c = \dot{z}_{C_0} \)).

Указанные уравнения можно также получить исходя из уравнений Лагранжа второго рода.
Обозначим координаты центра инерции твердого тела через \( x_c \), \( y_c \),
\( z_c \) и примем их за обобщенные координаты: \( q_1 = x_c \),
\( q_2 = y_c \), \( q_3 = z_c \).

Поступательное движение тела полностью определяется движением его центра
инерции, а поэтому число степеней свободы тела равно трем (\( s = 3 \)), его
кинетическая энергия \( \ds T = \frac{mv_c^2}{2} = \frac{1}{2}m(\dot{x}_c^2 +
\dot{y}_c^2 + \dot{z}_c^2) \) и уравнения Лагранжа второго рода будут иметь вид:
\[
    \der{}{t}\left( \pder{T}{\dot{q}_j} \right) - \pder{T}{q_j} = Q_j, \quad
    (j = 1,\ 2,\ 3).
\]
Левые части уравнений соответственно равны \( m\ddot{x}_c \), \( m\ddot{y}_c \),
\( m\ddot{z}_c \).

Далее, определяя \( \delta A = R\cdot\delta r_c = R_x\cdot\delta x_c +
R_y\cdot\delta y_c + R_z\cdot\delta z_c \), найдем обобщенные силы:
\[
    Q_1 = R_x = \sum F^e_{kx}, \quad Q_2 = R_y = \sum F^e_{ky}, \quad
    Q_3 = R_z = \sum F^e_{kz}.
\]
Откуда получаем: \( m\ddot{x}_c = \sum F^e_{kx} \),
\( m\ddot{y}_c = \sum F^e_{ky} \), \( m\ddot{z}_c = \sum F^e_{kz} \).
\newpage
