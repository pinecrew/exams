\question{Определение скорости и ускорения точки в полярной системе координат.
Дифференциальные уравнения движения материальной точки в цилиндрической
системе координат. Количество движения, кинетический момент и кинетическая
энергия материальной точки в цилиндрической системе координат.}

Мгновенной скоростью точки называется предел отношения вектора перемещения
точки к промежутку времени, за который произошло это перемещение, когда этот
промежуток времени стремится в нулю, то есть \( \ds \vec{v} = \lim_{\D t\to 0}
\Der{\vec{r}}{t} = \der{\vec{r}}{t} = \dot{\vec{r}} \). Скорость направлена по
касательной к траектории в сторону движения точки.

Рассмотрим движение заданное в полярных координатах, то есть пусть даны как
функции времени полярный радиус \( r = r(t) \) и угол \( \phi = \phi (t) \)
определяющие положение точки. Введем единичные вектора: \( \vec{n} \),
направленный по радиус-вектору в сторону возрастания \( \vec{r} \), и
\( \vec{p} \), повернутый относительно \( \vec{n} \) на угол \( \pi/2 \) в
сторону возрастания угла \( \phi \). Представим \( \vec{n} \) и \( \vec{p} \)
через векторы \( \vec{i} \) и \( \vec{j} \) декартовых координатных осей:
\[
    \vec{n} = \vec{i}\cos\phi + \vec{j}\sin\phi, \quad
    \vec{p} = \vec{i}\cos\left(\phi + \frac{\pi}{2}\right) + \vec{j}\sin\left(
    \phi + \frac{\pi}{2}\right) = -\vec{i}\sin\phi + \vec{j}\cos\phi.
\]

Продифференцируем по времени:
\[
    \der{\vec{n}}{t} = (-\vec{i}\sin\phi + \vec{j}\cos\phi)\dot{\phi} =
    \dot{\phi}\vec{p}, \quad \der{\vec{p}}{t} = -(\vec{i}\cos\phi +
    \vec{j}\sin\phi)\dot{\phi} = -\dot{\phi}\vec{n}.
\]

Радиус-вектор \( \vec{r} \) можно представить в виде \( \vec{r} = r\vec{n} \).
При движении точки меняются как модуль, так и направление радиус-вектора,
следовательно, и \( r \), и \( \vec{n} \) являются функциями времени. Тогда:
\[
    \vec{v} = \der{\vec{r}}{t} = \der{}{t}(r\vec{n}) = \der{r}{t}\vec{n} +
    r\der{\vec{n}}{t} = \der{r}{t}\vec{n} + r\der{\phi}{t}\vec{p}.
\]
Полученная формула дает разложение вектора скорости на две взаимно
перпендикулярные составляющие: радиальную \( v_r = \dot{r} \) и поперечную
\( v_p = r\dot{\phi} \).

Модуль скорости находится по формуле: \( v = \sqrt{v_r^2 + v_p^2} =
\sqrt{\dot{r}^2 + r^2\dot{\phi}^2} \).

Ускорением \( \vec{a} \) точки в данный момент времени называется предел
отношения приращения скорости к приращению времени при условии, что последнее
стремится к нулю, то есть \( \ds \vec{a} = \lim_{\D t\to 0} \Der{\vec{v}}{t} =
\der{\vec{v}}{t} = \dot{\vec{v}} = \ddot{\vec{r}} \).
 
В полярных координатах:
\[
    \vec{a} = \der{\vec{v}}{t} = \dder{r}{t}\vec{n} + \der{r}{t}\der{\phi}{t}
    \vec{p} + \dder{\phi}{t}\vec{p} + \vec{r}\der{\phi}{t}\der{\vec{p}}{t} =
    (\ddot{r}-r\dot{\phi}^2)\vec{n} + (r\ddot{\phi}+2\dot{r}\dot{\phi})\vec{p}.
\]
Получаем проекции ускорения на радиальное и поперечное направления:
\( a_r = \ddot{r}-r\dot{\phi}^2 \), \( a_p = r\ddot{\phi}+2\dot{r}\dot{\phi} \),
тогда модуль ускорения равен: \( a = \sqrt{a_r^2 + a_p^2} \).

Положение материальной точки \( M \) в инерциальной системе отсчета будем
определять ее радиус-вектором \( \vec{r} \). Сила \( \vec{F} \), действующая на
точку, может зависеть от положения точки, то есть от радиус-вектора, скорости
точки и времени. Следовательно в общем случае \( \vec{F} \) и основное уравнение
динамики точки можно записать в виде: \( \ds m\dder{\vec{r}}{t} =
\vec{F}\left(\vec{r}, \der{\vec{r}}{t}, t\right) \).
Это уравнение называется дифференциальным уравнением движения материальной точки
в векторной форме.

\subquestion{Количество движения материальной точки}
Перепишем основное уравнение динамики в виде
\( \ds \der{}{t}(m\vec{v}) = \vec{F} \) или \( d(m\vec{v}) = \vec{F}\,dt \).
Вектор \( \vec{q} \), равный произведению массы точки на ее скорость, называется
количеством движения материальной точки. Произведение силы на элементарный
промежуток времени ее действия называется элементарным импульсом силы. Последнее
уравнение выражает теорему об изменении количества движения материальной точки в
дифференциальной форме: элементарное изменение количества движения материальной
точки равно элементарному импульсу силы, приложенной к этой точке.

\subquestion{Кинетический момент (момент количества движения)}
Векторно умножим основное уравнение динамики на радиус-вектор точки
\( \vec{r} \), определяющий положение материальной точки относительно какой-либо
точки \( O \), которую будем называет центром:
\( \vec{r}\times m\vec{a} = \vec{r}\times\vec{F} \).

Преобразуем левую часть: \( \ds \vec{r}\times m\vec{a} = \vec{r}\times
m\der{\vec{v}}{t} = \der{}{t}(\vec{r}\times m\vec{v}) - \der{\vec{r}}{t}\times
m\vec{v} = \der{}{t}(\vec{r}\times m\vec{v}) \).

Тогда уравнение:
\( \ds \der{}{t}(\vec{r}\times m\vec{v}) = \vec{r}\times\vec{F} \).

Вектор \( \vec{K}_O = \vec{r}\times m\vec{v} \) называется моментом количества
движения материальной точки относительно центра (точки \( O \)).

\subquestion{Кинетическая энергия}
Запишем основное уравнение динамики материальной точки:
\( m\vec{a} = \vec{F} \), где \( \vec{F} \) -- равнодействующая всех сил,
приложенных к материальной точке. Умножим обе части этого равенства скалярно на
дифференциал радиуса-вектора \( d\vec{r} \):
\( \ds m\der{\vec{v}}{t}\cdot\d\vec{r} = \vec{F}\cdot\d\vec{r} \).

В правой части стоит элементарная работа \( dA \), левую часть представим в
виде:
\[
    m\der{\vec{v}}{t}\cdot\vec{r} = m\vec{v}\cdot\d\vec{v} =
    d\left(\frac{mv^2}{2}\right).
\]
 
Тогда равенство принимает вид: \( \ds d\left( \frac{mv^2}{2}\right) = dA \).

Половина произведения массы точки на квадрат ее скорости называется кинетической
энергией материальной точки.

\newpage
