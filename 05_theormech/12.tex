\question{Основные законы (теоремы) механики для материальной точки. Закон
сохранения импульса. Теорема об изменении количества движения.}

Основными теоремами механики материальной точки являются:
\begin{enumerate}
    \item теорема об изменении количества движения (и соответствующий ей закон
    сохранения импульса);
    \item теорема об изменении момента количества движения (и соответствующий ей
    закон сохранения момента импульса);
    \item теорема об изменении кинетической энергии.
\end{enumerate}

\subquestion{Теорема об изменении количества движения}

Количеством движения (импульсом) материальной точки
называется векторная величина, равная произведению её массы на скорость:
\( \vec{q} = m\vec{v} \).

Теорема об изменении количества движения является, по сути II законом Ньютона:
\[
    \dot{\vec{q}} = \vec{F}.
\]
Это дифференциальный вид теоремы об изменении количества движения.
Проинтегрировав полученное равенство по времени, имеем \( \ds \vec{q}_1 -
\vec{q}_0 = \int\limits_0^{t_1} \vec{F}\d t \), называемое интегральным видом
теоремы об изменении количества движения, где величину
\[
    \int\limits_0^{t_1} \vec{F}\d t
\]
называют импульсом силы.

\subquestion{Закон сохранения импульса материальной точки}

Отдельно можно рассмотреть случай \( \vec{F} = 0 \), при этом \( \dot{\vec{q}} =
0 \) или \( \vec{q} = \const \). То есть, если на материальную точку не
действует никаких сил, либо их действие скомпенсировано, то её количество
движения не изменяется (точка движется равномерно и прямолинейно).

\newpage % ---------------------------------------------------------------------