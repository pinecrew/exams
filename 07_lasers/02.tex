\question{Взаимодействие лазерного излучения с веществом}

Обращаясь к проблеме взаимодействия лазерного излучения с веществом, сначала 
необходимо в общих чертах охарактеризовать как излучение, так и вещество.

Если обратиться к лазерному излучению, то в первом приближении можно 
использовать для него привычный термин <<свет>>. Действительно, в настоящее 
время лазерный эффект осуществлен в диапазоне частот от ближнего 
ультрафиолетового до ближнего инфракрасного, то есть в световом диапазоне 
частот.

Если обратиться к веществу, то здесь диапазон весьма широк, от 
микроскопических объектов (атомы, молекулы) до макроскопических 
конденсированных прозрачных и непрозрачных сред (газы, плазма, жидкость, 
твёрдые тела).   

Процесс взаимодействия лазерного излучения с веществом на микроскопическом 
уровне представляет интерес с нескольких точек зрения. Во-первых, 
изолированный атом представляет собой относительно очень простой объект. 
Взаимодействие излучения с таким объектом можно достаточно строго описать 
аналитически и тем самым получить основные закономерности взаимодействия в 
форме, доступной для анализа. Во-вторых, изолированный атом представляет собой 
адекватную модель большого класса реальных сред -- разреженных газов. Наконец, 
в-третьих, закономерности, установленные для случая взаимодействия на 
микроскопическом уровне, существенно определяют взаимодействие излучения с 
плотными газами, плазмой, жидкостями и твёрдыми телами.

Если исходить из основного предположения, что среда прозрачна, то, очевидно, 
надо пот терминов взаимодействие иметь в виду процесс распространения 
излучения в среде. Основные законы распространения света в прозрачных средах, 
справедливые в рамках линейной оптики. Это закон прямолинейного 
распространения света; закон независимости световых пучков; закон отражения и 
преломления на границе различных сред; законы поглощения Бугера и Бера. 

Из курса оптики известно, что процесс взаимодействия света с непрозрачными 
средами сводиться к отражению и рассеянию света поверхностью и его поглощению 
в очень тонком поверхностном слое. В виду относительно малой интенсивности 
обычного, некогерентного света его поглощение в поверхностном слое не приводит 
к каким-либо интересным наблюдаемым эффектам, поэтому основное внимание всегда 
обращается на отражение и рассеяние света. Когда идёт речь о лазерном 
излучении, возникает качественно отличная ситуация. Исключительно большая 
интенсивность лазерного излучения обуславливает большое число разнообразных 
эффектов, связанных именно с поглощением излучения веществом. Под действием 
лазерного излучения твёрдые, непрозрачные тела -- например, металлы -- 
нагреваются до высокой температуры, расплавляются, испаряются, а их пары 
ионизуются, образуя плазму. Именно эти процессы представляют наибольший 
интерес, как для фундаментальных исследований, так и для практических 
приложений, из которых можно упомянуть столь крупные направления, как лазерный 
термоядерный синтез и лазерную обработку металлов.  