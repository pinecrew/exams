\question{Лазеры на жидких веществах и красителях}
Генерация осуществляется в импульсном и непрерывном режимах на переходах между 
уровнями возбуждённого и основного состояний сложных молекул органических 
красителей. Обычно используются разбавленные растворы красителей. Инверсия 
достигается по четырёхуровневой схеме оптической накачки. Излучение накачки в 
процессе синглет-синглетного поглощения заселяет колебательно-вращательные 
состояния верхнего терма. При этом в соответствии с принципом Франка-Кондона в 
случае сдвинутых равновесных конфигураций возбуждаются высшие колебательные уровни.

Внутри возбуждённого терма происходит быстрая безызлучательная релаксация, и 
энергия возбуждения переходит на нижние колебательные уровни этого терма. С 
нижних уровней возбуждённого терма происходит переход на верхние колебательные 
уровни основного терма. Таким образом, энергия излучённого фотона меньше 
энергии поглощенного. Дальнейшая термализация основного терма обеспечивает 
четырёхуровневый характер цикла оптической накачки.

Непрерывный спектр электронных термов является результатом наложения многих 
близколежащих колебательных состояний. При настройке резонатора на 
определённую частоту приводит к тому, что верхний терм опустошается именно на 
этой частоте. Высокая скорость внутритермовой релаксации приводит к тому,что в 
одночастотное излучение перекачивается вся энергия, накопленная возбуждённым 
термом.

Лазеры на красителях работают в интервале длин волн от ближнего ИК до ближнего 
УФ излучений. Плавная перестройка длины волны, являющаяся характерной 
особенностью этих лазеров, достигается в диапазонах шириной в несколько 
десятков нанометров при монохроматичности до нескольких мегагерц.