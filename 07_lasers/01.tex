\question{Предмет квантовой электроники. История возникновения и развития 
квантовой электроники. Современные достижения и возможности квантовой 
электроники.}

\subquestion{Предмет квантовой электроники}

\emph{Квантовая электроника} -- это область физики, изучающая методы усиления 
и генерации электромагнитного излучения путём использования эффекта 
вынужденного излучения в термодинамически неравновесных квантовых системах, 
свойства получаемых таким образом генераторов и усилителей и их применения.

Наука о мазерах и лазерах.

\subquestion{История возникновения и развития}

\begin{enumerate}
	\item[1905 г.] Эйнштейн пришёл к гипотезе световых квантов
	\item[1916 г.] Эйнштейн вывел формулу Планка, постулировав вынужденное 
		излучение
	\item[1924 г.] Бозе и Эйнштейн получили статистику Бозе-Эйнштейна, которой 
		подчиняются фотоны
	\item[1927 г.] Дирак строго обосновал существование вынужденного излучения 
		и его свойства
\end{enumerate}

\subquestion{Фундаментальные физические предпосылки создания квантовой 
электроники}

\begin{enumerate}
	\item[1954 г.] -- были даны непосредственные теоретические основы 
		квантовой электроники и создан первый прибор -- пучковый аммиачный 
		мазер
	\item[1955 г.] -- трёхуровневый метод накачки
	\item[1958 г.] -- открытый резонатор
	\item[1960 г.] -- первый лазер (рубиновый, гелий-неоновый)
\end{enumerate}

\subquestion{Современные достижения и возможности квантовой электроники}