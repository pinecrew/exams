\question{Интенсивность выходного излучения. Оптимальный коэффициент
прозрачности полупрозрачного зеркала резонатора}

Интенсивность излучения на выходе лазера определяется плотностью фотонов в
резонаторе, движущихся в сторону полупрозрачного зеркала, и прозрачностью
выходного зеркала и составляет
\[
  I_\text{вых} = h\nu n_p c\xi / 2,
\]
где \( n_p \)~-- концентрация фотонов, \( h\nu \)~-- энергия фотонов,
\( c \)~-- скорость света, \( \xi \)~-- коэффициент прозрачности.

Учитывая то, что коэффициент усиления может быть выражен следующим образом:\\
\( K = K_0 / (1 + I / I_s) \) и то, что интенсивность монохроматического света
можно выразить так: \( I = h\nu n_p c \), то
\[
  I_\text{вых} = \frac{I_s \xi}{2} \left( \frac{K_0}{K} - 1 \right).
\]

Так как коэффициент усиления среды в резонаторе в условиях стационарной
генерации равен \( K_\text{п} \), определяемому из формулы \eqref{eq12.1}, то

\begin{equation}
  I = \frac{I_s \xi}{2} \left( \frac{2K_0 L_a}{\ln\left[ (1 - \chi - \xi)^{-1}
    \right]} - 1\right).
  \label{eq_1.91}
\end{equation}

Считая \( \chi \) и \( \xi \) малыми величинами, можно избавиться от логарифма:
\[
  I_\text{вых} = \frac{I_s \xi}{2} \left( \frac{2K_0 L_a}{\xi + \chi} - 1
    \right).
\]

Как видно из последних соотношений, при \( \xi \ll \chi \) интенсивность
выходного излучения линейно растет с ростом прозрачности
\( I_\text{вых} \sim \xi \), а при \( \xi \gg \chi \)~-- падает. Таким образом,
существует оптимальная прозрачность \( \xi_\text{опт} \). В случае малого
усиления света за проход (\( 2K_0 L_a \ll 1 \)) для \( \xi_\text{опт} \) можно
получить аналитическое выражение, приравняв производную от \( I_\text{вых} \) по
\( \xi \) нулю и решив полученное уравнение
\[
    (\xi_\text{опт} + \chi)^2 = 2K_0 L_a\chi, \quad
    \xi_\text{опт} = \sqrt{2K_0 L_a\chi} - \chi.
\]
