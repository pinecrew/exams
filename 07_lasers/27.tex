\question{Газовые лазеры. Газодинамические лазеры}
Источником энергии излучения служит тепловая энергия равновесно нагретого 
молекулярного газа. Инверсия образуется при резком охлаждении газа за счёт 
процессов колебательной релаксации, идущих с различной скоростью для различных 
колебательных мод многоатомной молекулы или различных компонент газовой смеси. 
Скорость охлаждения должна быть достаточно велика, чтобы релаксационные 
процессы не успели опустошить верхний лазерный уровень. Требуемое быстрое 
охлаждение больших массовых потоков газа осуществляется газодинамически при 
сверхзвуковом истечении сжатого и нагретого газа в вакуум.

В случае газодинамического \( CO_2 \)-лазера для нагрева может быть 
использовано сжигание органического топлива. Тогда получающийся при сгорании 
водяной пар можно использовать в качестве буферного газа, вместо 
дорогостоящего гелия.

Этот тип лазеров работает как в импульсном, так и в непрерывном режимах. По 
сути, они являются тепловыми машинами, в которых тепловая энергия 
преобразуется в энергию когерентного электромагнитного излучения.
