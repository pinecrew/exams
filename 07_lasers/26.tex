\question{Газовые лазеры. Химические лазеры}
Инверсия создаётся при неравновесном распределении энергии между внутренними 
степенями свободы продуктов экзотермических химических реакций за счёт 
энергии, выделившейся в процессе реакции, как правило на переходах между 
колебательными уровнями молекул в газовой фазе. Генерация реализована на 
колебательно-вращательных переходах двухатомных молекул галогеноводородных 
соединений, получаемых в ходе реакций замещения типа
\[
 	\mathrm{A + BC \rightarrow AB^* + C.}
\]

Для снижения доли энергии, идущей на инициирование такого химического 
процесса, используются цепные самоподдерживающиеся реакции. Чем больше длина 
цепи реакции, тем меньшую роль играет энергия инициирующего воздействия.

Возможны как импульсный, так и непрерывный режимы работы.В импульсном режиме 
УФ фотодиссоциация или электронно-пучковый радиолиз инициируют цепную реакцию, 
протекающую достаточно быстро, с тем чтобы релаксационные процессы не успели 
сбросить инверсию. Непрерывный процесс возможен при сливании газов и удалении 
продуктов реакции прокачкой газовой смеси. При сливании взаимно нестабильных 
реагентов и быстром удалении продуктов реакции возможен чисто химический лазер 
без инициирования.

Для химических лазеров характерны следующие длины волн: \( HF \) -- 2,7 мкм,
\( HCl \) -- 3,7 мкм, \( HBr \) -- 4,2 мкм и \( DF \) -- 4,3 мкм.

Достоинством этого типа лазеров является возможность получения инверсии в 
больших объёмах и при больших массовых расходах активного вещества, а также 
отсутствие необходимости заметных затрат энергии на создание инверсной 
заселённости. Характерной чертой этих лазеров является такой химический 
механизм создания инверсии, при котором энергия излучения превышает энергию 
инициирования химической реакции.
