\question{Рационализм в философии XIX -- начала XX вв. (Идеи позитивизма и марксизма).}

\subquestion{Позитивизм}
Позитивизм~-- философское направление, настаивающее на том, что все положительное (позитивное)
философское знание сводится к содержанию отдельных специальных наук и обобщению их достижений. Таким
образом, философия как особая наука, претендующая на содержательное исследование проблемы
соотношения сознания и реальности, другие мировоззренческие проблемы не имеет права на
существование. <<Наука сама себе философия>>~-- центральный тезис позитивизма.

Возникновение позитивизма связано с философской доктриной философа Огюста Конта, которую он изложил
в труде <<Курс позитивной философии>>. Предназначение своей философии, которую Конт назвал
<<синтетической>>, он видел в <<признании всех явлений подчиненными неизменным естественным законам,
открытие и низведение числа которых до минимума и составляет цель всех наших усилий>>. Философия для
Конта~-- некая общая наука, раскрывающая связь отдельных наук на основе проведенной им классификации
наук. Причем в классификации Конт руководствуется достаточно естественным принципом рассмотрения
наук по мере их усложнения. В его классификации возникает следующая цепочка наук: механика с
математикой, науки о неорганических телах (астрономия, физика, химия), науки об <<органических
телах>> (биология, физиология и социология). Конт отвергает деление философских школ на
материалистов и идеалистов как проявление <<метафизики>>, то есть традиционной, спекулятивной
философии. По мнению Конта, наука не объясняет, а лишь описывает изучаемые явления, отвечает на
вопросы <<как>>, а не <<почему>>.

По сути, вся <<положительная>> философия Конта состоит из набора обобщений, не отличающихся от
обобщений частных наук, и имеет мало общего с традиционным пониманием предназначения философии и ее
проблематики.

Другим видным представителем позитивизма являются Герберт Спенсер и Джон Стюарт Милль.

\subquestion{Марксизм}
Основателями марксизма являются Карл Маркс и Фридрих Энгельс. 

В центре теории марксизма~-- концепция исторического развития, идея объективных
социально-экономических закономерностей. В создании теории Маркс и Энгельс опирались на эмпирические
факты и достижения общественных наук. Они считали, что законы развития общества объективны так же,
как законы природы. Значение своих работ они видели в раскрытии социально-экономических
закономерностей, знание которых позволит людям овладеть своей материальной и общественной жизнью.
Осознанно управляя условиями своей жизни, люди станут свободными.

В основе общественной жизни лежит материальное производство, главной характеристикой которого
является способ производства. В соответствии с определенными типами способа производства в марксизме
выделяют ряд форм общественного устройства, которые получили название формаций: племенную, античную,
феодальную, капиталистическую и коммунистическую. Способ производства представляет собой единство
производительных сил (орудий труда, технологий, сырья, капитала) и складывающихся на их основе
производственных отношений. Основными производственными отношениями являются отношения собственности
(отношения к средствам производства) и отношения распределения материальных благ. На их основе
складывается социальная структура общества; способ производства, основанный на частной
собственности, обуславливает господство класса, владеющего средствами производства и эксплуатацию
неимущего класса. Производительные силы общества развиваются благодаря прогрессу науки и техники, и
рано или поздно вступают в противоречие с существующими производственными отношениями. Чтобы
материальное производство могло развиваться, это противоречие должно разрешаться в ходе социальной
революции. Когда существующие производственные отношения больше не способствуют развитию
производственных сил, а тормозят его, угнетенный класс оказывается в бедственном положении, вступает
в противоречие с другими классами и ведет борьбу за свое освобождение. В марксизме вся история
оказывается историей классовой борьбы. Результатом революции становится переход к новому способу
производства.

Изменение способа производства, установление новых экономических отношений первым делом отражается
на политическом устройстве общества, на законах. В воле государства и законах отражается интерес
господствующего класса, который стремится укрепить свое положение, узаконить свой экономический
интерес. Класс, обладающий политической властью, владеющий средствами материального производства,
владеет также средствами духовного производства. Он стремится представить свой интерес в качестве
интереса всего общества, поэтому распространяет соответствующие идеи, порождает новую идеологию. В
своих работах Маркс и Энгельс называют идеологиями мораль, религию, философию~-- они не имеют
самостоятельного существования и, на самом деле, выражают экономический интерес господствующего
класса. Таким образом, вся совокупность политических и правовых отношений, духовной культуры и
соответствующих им идей и представлений, называемая надстройкой, полностью определяется способом
производства, который составляет экономический базис общества. Общественное сознание, духовная жизнь
общества, представляет собой отражение общественного бытия, экономической жизни общества.

Свою историческую концепцию Маркс и Энгельс называли материалистическим пониманием истории.
Центральной в ней является представление о том, что <<способ производства материальной жизни
обуславливает социальный, политический и духовный процессы жизни вообще>>.

Маркс во многих работах жестко критикует капиталистический строй: в таком обществе человек не имеет
возможности получать удовлетворение от труда, который является принудительным, а не свободным; от
общения, которое сводится к бессердечным товарно-денежным отношениям. Чтобы преодолеть человеческое
отчуждение, создать лучшее, справедливое общество, идеалом которого в марксизме выступает
коммунистическое общество, необходимо уничтожить существующие общественные отношения.

Маркс и Энгельс, основываясь на открытых ими закономерностях исторического развития, пытались
обосновать необходимость революции. Они считали, что противоречия капиталистического общества будут
только усугубляться в силу роста производительных сил. Угнетенный класс пролетариата, порожденный
самим же капитализмом, будет расти, а его положение будет становиться все более невыносимым.
Пролетариат должен будет осознать свое бедственное положение, объединиться и выступить против
существующего порядка. Социалистическая революция будет направлена, прежде всего, на захват
политической власти, в результате чего пролетариат станет господствующим классом и возникнет
социалистическое государство. Власть позволит пролетариату принять меры по уничтожению частной
собственности. В результате исчезнут классы, социальное неравенство, отпадет необходимость в
политической власти, деньгах – на смену социалистическому государству придет коммунистическое
общество. Оно представляется как ассоциация, в которой человек сможет свободно трудиться на благо
общества, реализуя свои способности, вступать в солидарные отношения с другими людьми, получать
удовлетворение от деятельности и общения.
