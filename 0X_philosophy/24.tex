\question{Человек как предмет философского анализа.}

Человек -- одна из основных проблем философской рефлексии. Рассмотрение человека как особой философской темы 
обусловлено потребностью в целостном подходе к его изучению. Потребность эта возникает и расширяется по мере 
того, как интерес к человеку становится универсальной тенденцией развития различных конкретных наук: 
политэкономии и социологии, биологии и медицины, астрономии и географии, этнографии и антропологии, 
лингвистики и культурологи и т.д. В искусстве все большее место стала занимать идея преломления природных и 
социальных явлений сквозь призму видения их человеком. Повышение интереса к философскому анализу проблемы 
человека продиктовано сегодня новым этапом научно-технической революции (и её влияния на индивида), 
развитием мирового сообщества, экологической ситуацией и многими другими проблемами.

Сложность философского определения человека состоит в невозможности однозначного соотнесения его с каким-либо 
родовым понятием (например, природа, Бог или общество), поскольку человек -- это всегда одновременно 
микрокосм, микротеос и микросоциум. Поэтому философское постижение человека разворачивается не просто через 
реконструкцию его сущностных характеристик, но через осмысление его бытия в мире, понимания собственно 
человеческого мира.

Проблема человека в науке отличается своеобразным редукционизмом: через связь человека с неким феноменом 
объясняется все человеческое бытие в целом. Наука максимально отвлекается от решения важнейших 
мировоззренческих проблем: она ориентирована лишь на эмпирический уровень человеческого бытия. Философия же 
стремиться абстрагироваться от действительности, чтобы понять не только то, что есть, но и как должно быть.

В истории философии человек понимался традиционно в единстве таких его основных модусов, как тело, душа и 
дух. Тело -- это физическая субстанция человеческой жизни, выступающая как элемент природы, в соответствии с 
интерпретацией которой можно говорить об основных образах тела в истории философии и науки (микрокосм, 
механизм и организм). Одновременно с этим человеческое тело определяется не только через его биологические 
особенности, но и через особый спектр таких исключительно человеческих чувств и состояний, как совесть, стыд, 
смех, плач и т.п. 

Душа рассматривается как интегративное начало, промежуточное звено, соединяющее тело и дух, придающее 
человеку целостность. Для современной философии душа – наиболее сложная и противоречивая тема, 
рассматриваемая в двух основных ракурсах:
\begin{itemize}
    \item во-первых, как жизненный центр тела, являющийся той силой, которая, будучи сама бессмертной, 
        очерчивает срок телесного существования (в связи с признанием существования или несуществования души 
        в философии возникали вопросы о смерти и бессмертии, бытии и небытии);
    \item во-вторых, как индивидуализирующее свойство человека в обществе, описываемое в философии через 
        проблемы свободы воли, творчества, рока и судьбы.
\end{itemize}

Понятие душа тесно связано в человеческом существовании и его осмыслении с понятием дух. Дух воплощает в себе 
фундаментальную идею <<человечности>> как таковой. Он выступает как родовая человеческая способность, 
соотносящаяся с разумом, сознанием и социальностью. Вместе с тем, в понятии духа отражается не только феномен 
<<духовности>>, как интегративного начала культуры и общества, но и личностные характеристики отдельного 
человека, где личное характеризуется через воплощение социально значимых качеств.

Однако, человека нельзя упрощенно представить как диаду (тело –- дух) или триаду (тело –- дух –- душа). 
Человек -- это практически всегда исключение из общего правила, уникальная целостность, где в индивидуальном 
личном опыте достаточно трудно дифференцировать телесный, душевный и духовный уровни. 