\question{Категория бытия в философии.}

По гречески онтос. Придумал термин Парменид, по его мнению: <<Бытие есть, небытия нет!>>

Бытие -- наиболее общая категория, обозначающая всё, что существует (даже мысли). Чтобы уточнить это понятие ищут то общее, что всё объединяет.

Вводят две категории: 
\begin{itemize}
	\item субстрат (то из чего всё состоит);
	\item субстанция (самостоятельная сущность, которой ничего не требуется для своего существования кроме себя? основа сущего).
\end{itemize}

Учение в котором в качестве основы выступает одно начало -- монизм, два -- дуализм (Декарт (материальное и идеальное (дух)) и Кант) и плюрализм (Лейбниц, субстанция -- каждая монада по-отдельности).

Монизм бывает:
\begin{itemize}
	\item идеалистический (у Пифагора в качестве основы выступают числа, у Платона -- эйдосы, у Аристотеля -- формы, у Гегеля роль субстанции выполняет Абсолютная идея, бог в религиозных картинах мира, например, у Тейяра-де-Нардена или Спинозы);
	\item материалистический (Маркс-Энгельс-Ленин: материя -- субстанция).
\end{itemize}

Формы бытия:
\begin{itemize}
	\item материальное -- явления и предметы окружающего мира;
	\item идеальное;
	\item человеческое -- совокупность материального и идеального бытия связанного с человеком;
	\item ноуменальное -- существующее независимо от наблюдателей;
	\item феноменальное -- то, каким видит бытие субъект. 
\end{itemize}

Противоположная категория -- небытие. Есть одна особенность -- философы не признают четырёхмерную действительность, поэтому к небытию относятся не рождённые и умершие люди, цивилизации, города, не созданные вещи (Парменид бы перевернулся в гробу).