\question{Специфика научного познания, его формы и методы. Проблема рациональности.}

Наука -- производство знаний, постижение истины и открытие объективных законов действительности.

Специфика научного познания:
\begin{itemize}
	\item поиск объективных законов действительности
	\item прогностическая функция
	\item системность
	\item требуются специальные знания и терминология
	\item требуются оборудование и знание методов
	\item строгая доказательность и обоснованность
\end{itemize}

Различают два уровня научного познания:
\begin{itemize}
	\item эмпирический (наблюдение, эксперимент, сравнение, измерение, описание);
	\item теоретический (выдвижение, построение и разработку научных гипотез и теорий; формулирование законов; выведение логических следствий из законов; сопоставление друг с другом различных гипотез и теорий, теоретическое моделирование, а также процедуры объяснения, предсказания и обобщения).
\end{itemize}

Научная рациональность – это соотношение познания с образцами, логическими и методологическими нормами.

Научная рациональность – это вид деятельности, направленной на получение нового научного знания, поиска истины, отличается строгими способами доказательства.

\begin{enumerate}
	\item Позитивистская мысль. Решение проблемы научной рациональности напрямую связано с поиском «истиной» науки.
	\item Неопозитивистская идея научной рациональности – заключалась в сводимости применения теоретического уровня знания к эмпирическому.
\end{enumerate}

Виды научной рациональности:
\begin{enumerate}
	\item Логико-математическая: идеальная предметность, формальная доказательность.
	\item Естественнонаучная: эмпирическая предметность, предполагается возможность бесконечной воспроизводимости результатов наблюдений, частичная логическая доказуемость, опытная проверяемость;
	\item Инженерно-техническая: практическая эффективность, предметность;
	\item Социально-гуманитарная: ценностная предметность, культурологическая обоснованность, адаптивная полезность, рефлексивность.
\end{enumerate}

В современной философии существуют 2 основные линии осмысления проблемы рациональности: 
\begin{itemize}
	\item сциентизм (Конт) акцент делается на науке и поиске строгих средств систематизации знания; рациональность отождествляется с научной рациональностью в ее классической форме; cциентизм представлен позитивизмом, неопозитивизмом и постпозитивизмом;
	\item антисциентизм (Шеллинг, Шопенгауер, Ницше, Гуссерль, Хайдеггер, Ясперс, Камю) интерпретация проблемы рациональности связана с философией жизни, экзистенциализмом, философской антропологией; в рамках антисциентизма акцент делается на вненаучные способы постижения действительности, на спонтанность человеческого поведения и вторичность рассудка; наука отодвигается на второй план, ставится в один ряд с другими формами духовной культуры; крайний антисциентизм полностью отрицает ценность науки.
\end{itemize}