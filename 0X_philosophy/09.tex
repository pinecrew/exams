\question{Проблемы человека и познания: споры Сократа и софистов.}

Софисты, или мудрецы, современники Сократа -- это профессиональные риторы-преподаватели, философы, 
стремившиеся познать основы мироздания и изложить открывшиеся знания в пространных учениях. Главной темой 
этих учений было исследование первопричин бытия, его составляющих частей и движущих начал. Признаком 
неистинности такого рода философии Сократ считал коренные, неразрешимые противоречия её отдельных учений. 
<<Говоря об абсолютном и вечном... философы не только не сходятся между собою, но "подобно безумным" 
совершенно противоречат друг другу относительно того же самого предмета -- природы вещей>>. Видя эти тщетные 
попытки, Сократ приходит к выводу о принципиальной непознаваемости абсолютных истин. Но для внешнего 
человеческого это знание является закрытым: <<если бы мы знали законы природы, тайны мироздания, то сами были 
бы богами, ибо абсолютное знание свойственно только Богу, а не людям>>. Ограниченный человеческий разум 
неспособен вместить в себя всех вселенских тайн, и это знание может быть дано ему только путём откровения. 
Именно поэтому Сократ восстал против бесплодного умствования софистов. С точки зрения Сократа, вес 
философской мысли должен быть перенесен с недоступных <<дел божественных>> на <<дела человеческие>>, которые 
зависят от свободной воли человека и находятся в его власти. То, к чему в полной мере могут быть приложены 
добродетельные силы Души -- к теоретическому и практическому познанию значений таких понятий как право, 
справедливость, закон, государство, благочестие и мудрость софистами и риторами оставляется в стороне. Сократ 
же наоборот пытается подвигнуть своих слушателей к исканию истинных норм человеческих отношений посредством 
самоиспытания и самопознания. По Сократу истинно только то знание, которое может быть применено на практике и 
единственным способом философского делания может быть только личное благочестие. <<Поэтому-то Сократ, сводя 
добродетель к знанию блага, отрицал, чтобы этому знанию можно было обучать, чтобы его можно было преподавать 
за известную мзду, как это делали софисты>>.

Одним из основных разногласий Сократа и софистов стал вопрос о существовании объективной истины. Софисты были 
уверены в отсутствии истины вне человека и считали каждого человека вправе принимать за истину то, что 
подходит для конкретного индивида в конкретном случае в зависимости от личных склонностей, ситуации, выгоды и 
прочее. В вопросе истинности того или иного положения решающим становится субъективное мнение и произвол 
человека, <<меры всех вещей>>. Таким обращом в рассуждениях софистов <<о делах человеческих>> Сократ не 
находит ничего, кроме напущенной мнимой мудрости. По его мнению, без непреложных истинных начал, истинных 
норм человеческой деятельности -- теоретической и практической невозможна разумная и созидательная 
деятельность человека и какое-либо положительное развитие личности. Задача каждого -- найти всеобщие и 
объективные теоретические и практические, логические и этические нормы. К этому и был призван Сократ, как 
<<овод к коню>> приставленный к гражданам Афин, не давая сытой успокоенности взять верх, разрушая всякое 
мнимое знание и <<повивая>> рождение истины. Причем такой процесс возможен только при условии приложения 
конкретных усилий каждым конкретным человеком -- нахождение всеобщей истины, применимой к каждому из нас, 
ставится в прямую зависимость от личного благочестия. <<Не стыдно ли тебе, что ты заботишься о деньгах..., о 
славе и почестях, а о разумности, об истине и о душе своей, чтобы она была как можно лучше, не заботишься и 
не помышляешь?>> -- обращается Сократ к своим согражданам. Многозаботливости и рассеянности сил человека 
Сократ предлагает <<единое на потребу>> - заботу о самом себе, о <<самом дорогом>>, о своей душе. И кроме 
утверждения о собственном незнании, Сократ вошёл в историю и с проповедью самоиспытания: <<Познай самого 
себя>>, - призывал философ.