\question{Иррационализм в философии XIX века.}

С середины XIX возникает иррационализм -- направление в философии, настаивающее на ограниченности 
человеческого ума в постижении мира. Иррационализм предполагает существование областей миропонимания, 
недоступных разуму, и достижимых только через такие качества, как интуиция, чувство, инстинкт, откровения, 
вера и т. п.

Основоположником европейского иррационализма является Шопенгауэр. Мир, согласно Шопенгауэру, может 
обнаруживаться человеком и как воля, и как представление. Воля -- это абсолютное начало всякого бытия, некая 
космическая и биологическая по своей природе сила, сотворившая мир и человека. С появлением последнего 
возникает мир как представление, как человеческая картина. Человек является рабом воли, поскольку во всем 
служит не себе, а Абсолюту. Воля заставляет человека жить, каким бы бессмысленным ни было его существование. 
Она заманивает индивида призраками счастья и такими соблазнами, как, например, сексуальное наслаждение. На 
самом же деле человек имеет для воли лишь косвенное значение, так как служит средством для ее сохранения. У 
человека есть только один выход -- погасить в себе волю к жизни. По его мнению, каждый человек располагает 
тремя высшими благами жизни -- здоровьем, молодостью и свободой. Пока они есть, индивид их не осознает и не 
ценит, осознает же лишь в случае их утраты, поскольку эти блага, по Шопенгауэру, только отрицательные 
величины. 

Предшественниками иррационализма в философии были Ф. Г. Якоби, и, прежде всего, Г. В. Й. Шеллинг. Но, как 
утверждал Фридрих Энгельс, работа Шеллинга Философия откровения (1843) представляет собой <<первую попытку 
сделать из преклонения перед авторитетами, гностических фантазий и чувственной мистики свободную науку 
мышления>>.

Ключевым элементом иррационализм становится в философиях С.Кьеркегора, А. Шопенгауэра и Ф. Ницше. Влияние 
этих философов обнаруживается в самых различных направлениях философии (прежде всего немецкой), начиная с 
философии жизни, неогегельянства, экзистенциализма и рационализма вплоть до идеологии немецкого 
национал-социализма. Даже критический рационализм К. Поппера, часто называемый автором самой рациональной 
философией, характеризовался как иррационализм (в частности, австралийским философом Д. Стоувом).

Необходимо мыслить дислогично, соответственно, иррационально, чтоб познать иррациональное. Логика -- 
рациональный способ познания категорий бытия и небытия, можно мыслить (насколько это возможно), что 
иррациональный способ познания кроется в дислогичных методах.

Перейдём к одному из ярчайших представителей европейского философского иррационализма был немецкий мыслитель 
Фридрих Ницше. Жизнь Ницше рассматривал как <<волю к власти>>. Все живое, согласно философу, стремится к 
власти, неравенство же сил создает естественное разграничение. Жизнь -- это борьба всех против всех, в ней 
побеждает сильнейший. Насилие, согласно Ницше, есть чистое проявление прирожденной воли человека к власти.

Главную причину краха современной ему цивилизации философ видел в засилии интеллекта, в преобладании его над 
волей. Там, где интеллект возвышается над волей, она обречена на неминуемое разложение. Именно поэтому разум 
должен быть подчинен воле и работать как орудие власти.

Ницше пытался разорвать границы чисто теоретического познания и ввести в него в качестве регулятора 
практическую жизнь. Однако этот регулятор оказался не чем иным, как инстинктивной деятельностью, направляемой 
слепой иррациональной волей к власти.

Ницше одним из первых сказал о наступлении нигилизма, т.е. времени, когда христианский Бог утратил свою 
значимость для европейской культуры. Назначение отрезвленного нигилизмом европейского человека мыслитель 
видел в том, чтобы мужественно восторжествовать над остатками иллюзий.