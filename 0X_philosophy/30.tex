\question{Аксиология как раздел философии. Многообразие мира ценностей.}

Аксиология -- раздел философии, в котором изучаются ценности, и собственно система ценностей.

Историю аксиологии в истории философии традиционно принято начинать с Лотце, немецкого физиолога и философа, который  придал понятию <<ценность>> категориальный смысл, имеющий значение как для бытия, так и для познания.

Ценность – это особый вид реальности. Сама по себе она не существует, хотя и связана не только с человеком, но и с объективным миром. Мир полон ценностей – материальных (вещи, деньги, собственность), художественных (произведения искусства и литературы), природных (солнечный восход, моря, цветы, ландшафты), собственно человеческих (смех, красота глаз, мужественный поступок).

Ценность всегда и одновременно ценность чего-то (кого-то), и ценность для кого-то. Синоним ценности -- значимость.

Виды ценностей:
\begin{itemize}
	\item материальные, духовные
\end{itemize}

\begin{itemize}
	\item коллективные, личные
\end{itemize}

\begin{itemize}
	\item моральные, научные, эстетические, юридические, философские, религиозные, социальные, политические, экономические, финансовые, экологические
\end{itemize}

Подлинные ценности отличает критерий незаменимости. Их нельзя измерить деньгами.
