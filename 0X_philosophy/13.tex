\question{Эмпиризм и рационализм в философии Нового времени.}
\begin{itemize}
    \item Френсис Бэкон --- родоначальник традиций эмпиризма в Англии --- обращает главное внимание на изучение природы, выработку нового взгляда на науку и её классификацию, цели и методы исследования. К причинам, задерживающим умственный прогресс относит:
    \begin{itemize}
        \item ограниченность ума и органов чувств
        \item индивидуальные недостатки
        \item влияние особенностей социальной жизни
        \item склонность к вере в авторитеты
    \end{itemize}
    Критикует схоластику за понятия субстанции, скрытого качества и т. п., так как на них нельзя построить науку. Реально существуют лишь отдельные предметы и их отношения, поэтому в основе научного познания должно лежать изучение отдельно существующих вещей и индукция как метод анализа и обобщения опытных данных.
    \item Рене Декарт разрабатывает рационалистическую методологию, возражает против преувеличенных оценок роли чувственного опыта в познании, т.к. сущность вещей познаётся другим путём. Интуиция и дедукция --- основные компоненты метода Декарта. Интуиция сближается с врождёнными идеями: идее Бога, субстанции, движения и т.п. Вводит принцип радикального сомнения по отношению к человеческому познанию. Систематическое изложение философии должно начаться с некоторого интуитивно ясного представления, например, <<я мыслю>>. Отсюда получаем <<я мыслю, следовательно, существую>>. Достоверность представлений о вещах вне нас Декарт возлагает на бога, который не способен на обман.
    \item Томас Гоббс полагал, что процесс познания начинается с чувственного опыта и в человеческом уме нет ни одного понятия, которое не было бы порождено в органах чувств. На чувственном опыте основано мышление и язык, теоретические утверждения и наука.
    Что касается социально-политических взглядов, то Гоббс выделяет 2 состояния общества: естественное (человек человеку волк) и государственное. Переход к государственного происходит путём общественного договора.
    \item Джон Локк подверг критике концепцию врожденных идей. Он исходил из предположения, что врождённых идей нет и в разуме есть только то, что было раньше в ощущениях. Локк показывал, как идёт движение от простых идей к сложным.
    \item Барух Спиноза продолжил рационалистическую линию, считая, что в чувственном опыте много субъективного, неясного. Образцом рационального познания он считал математико-геометрическое, которое лишено субъективизма. Логический вывод опирается на некоторые интуитивные предположения, принятые в качестве исходных истин. Интуиция даёт нам в качестве исходного понятие субстанции, которая является причиной самой себя. Содержание субстанции выражается в её атрибутах, а для указания причинно-следственной связи вводится понятие модуса. Спиноза полагал, что в мире нет ничего случайного, так как всё предопределено субстанцией.
    \item Готфрид Вильгельм Лейбниц считал, что в основе всего лежит субстанция, которая понимается как множество монад. Бог сотворил монады --- некоторые духовные образования, и материю как внешнюю оболочку монад. В человеке есть господствующая монада, то, что называют душой. В ней происходит саморазвитие заложенного знания. Лейбниц полагал, что главные знания не эмпирические, а рациональные -- всеобщие и необходимые. Их содержание интуитивно и опирается на законы логики.
\end{itemize}