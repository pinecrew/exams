\question{Природа и общество. Глобальные проблемы современности.}

В самом широком значении природа~-- это все существующее в бесконечном многообразии своих
проявлений. В этом смысле понятие природы выступает синонимом понятий Вселенная, материя, мир.

Общество, с позиций материалистической философии, является закономерным результатом на пути
поступательного развития материи.

Научный подход в определении взаимосвязи природы и общества предполагает учитывать как их
единство, так и их различие. Общество~-- это качественно определенная часть природы,
характеризующаяся системой материальных и духовных отношений, основу которых составляет
материальное производство. Общество в своем развитии постоянно взаимодействует с природой и
развитие его немыслимо без естественных предпосылок, к числу которых относится географическая
среда и народонаселение.

Понятие географической среды охватывает не всю бесконечную природу, а ту ее часть, с которой
непосредственно или опосредованно соприкасается человек. Элементами географической среды
являются: рельеф, климат, почва, природные богатства земных недр, флора, фауна, естественные
пути сообщения и т.д. Географическая среда необходима для жизни общества и оказывает
непосредственное влияние на его развитие. Географическая среда не может определять развитие
общества и тем более политику государства, но она может ускорить или замедлить темпы развития
стран и народов.

Народонаселение~-- это совокупность людей, проживающих в пределах определенных территорий. Без
наличия определенного минимума людей невозможны производство и общественная жизнь в целом. В
XVII веке на основе теории трудовой стоимости возникла демографическая теория, согласно
которой рост народонаселения является тем основным фактором, который лежит в основе
исторического процесса и определяет физиономию общества на том или ином отрезке этого процесса.
Однако довольно скоро многие ученые того времени поменяли свою точку зрения: быстрый рост
народонаселения неизбежно приводит к росту социальных проблем: безработице, преступности,
голоду. В целом, рост народонаселения, его численность, плотность, размещение по территории~--
все это оказывает влияние на ход исторического развития, может ускорять или замедлять развитие
общества, но причиной его развития является противоречие между производительными силами и
производственными отношениями. 

Глобальные проблемы современности~-- это совокупность жизненно важных проблем, от решения
которых зависит дальнейший социальный прогресс и будующее человечества. ГПС~-- это
проблемы, которые затрагивают интересы всего человечества; если не будет найдено их
решение, то человечеству угрожает гибель или серьезный регресс в условиях самой жизни.
Решение ГПС возможно только совместными усилиями всех стран мира. 

К числу основных глобальных проблем современности относят следующие:
\begin{enumerate}
    \item проблема войны и мира: предотвращение мировой ядерной войны;
    \item экологическая проблема: предотвращение катастрофического загрязнения окружающей
        среды, явления "парникового эффекта", образования "озоновых дыр" и т.д.;
    \item ресурсная проблема: проблема продовольствия, промышленного сырья и источников
        энергии;
    \item проблема освоения мирового океана: в недрах морского дна запасы железа примерно
        одинаковы с запасами на суше, обнаружены огромные запасы нефти и газа, каменного
        угля и цветных металлов. Кроме того, мировой океан обладает огромным резервом для
        решения сырьевой и продовольственной проблем;
    \item проблема освоения космического пространства;
    \item проблема преодоления возрастающего разрыва в уровне экономического и культурного
        развития между развитыми индустриальными странами и развивающимися странами Азии,
        Африки, Латинской Америки, устранение экономической отсталости, ликвидации голода,
        нищеты и неграмотности. 
\end{enumerate}

Следует различать глобальные проблемы и проблемы универсальные: универсальные проблемы~--
это общие проблемы для всех стран, но для их решения не требуется усилий всех стран планеты.
