\question{Пространство и время: эволюция представлений, современный взгляд.}

Ну тут всё всем ясно, поэтому только основные вещи.

Свойства пространства:
\begin{itemize}
    \item размерность
    \item изотропность
    \item однородность
\end{itemize}

Основным свойством времени считается необратимость --- процессы протекают из настоящего в будущее.

В истории философии сложились 2 основные концепции пространства и времени:
\begin{enumerate}
    \item субстанциальная концепция: пространство и время как самостоятельные сущности, пространство --- абсолютная пустота, вместилище вещей и явлений, время --- бесконечный поток состояний и событий;
    \item реляционная концепция понимает пространство и время как систему отношений, которые не существуют сами по себе, а выступают как атрибуты вещей. В рамках этой концепции могут быть рассмотрены неклассические представления о пространстве-времени:
    \begin{itemize}
        \item СТО
        \item ОТО
    \end{itemize}
\end{enumerate}

А вообще, пространство --- это множество со структурой.