\question{Смысл и ценности человеческой жизни. Основные категории человеческого существования.}

Смысл жизни:
\begin{enumerate}
	\item Аристотель: достичь счастья, которое состоит в мышлении и познании;
	\item Эпикур: получение удовольствий;
	\item киники: достичь счастья, которое состоит в том, чтобы довольствоваться малым и не творить зла;
	\item христианство: спасти душу;
	\item буддизм: разорвать круг перерождений;
	\item Шопенгауэр: его нет, так как всё заранее предопределено Мировой Волей;
	\item Сартр, Кьеркегор: всё зависит от человека.
\end{enumerate}

Аксиология -- раздел философии, в котором изучаются ценности, и собственно система ценностей.

Историю аксиологии в истории философии традиционно принято начинать с Лотце, немецкого физиолога и философа, который  придал понятию <<ценность>> категориальный смысл, имеющий значение как для бытия, так и для познания.

Ценность – это особый вид реальности. Сама по себе она не существует, хотя и связана не только с человеком, но и с объективным миром. Мир полон ценностей – материальных (вещи, деньги, собственность), художественных (произведения искусства и литературы), природных (солнечный восход, моря, цветы, ландшафты), собственно человеческих (смех, красота глаз, мужественный поступок).

Ценность всегда и одновременно ценность чего-то (кого-то), и ценность для кого-то. Синоним ценности -- значимость.

Виды ценностей:
\begin{itemize}
	\item материальные
	\item духовные
\end{itemize}

\begin{itemize}
	\item коллективные
	\item личные
\end{itemize}

\begin{itemize}
	\item моральные
	\item научные
	\item эстетические  
	\item юридические
	\item философские 
	\item религиозные 
	\item социальные
	\item политические
	\item экономические 
	\item финансовые
	\item экологические
\end{itemize}

Подлинные ценности отличает критерий незаменимости. Их нельзя измерить деньгами.

К основным категориям человеческого бытия относятся:
\begin{itemize}
	\item свобода
	\item смысл жизни
	\item творчество  
	\item любовь
	\item счастье 
	\item вера 
	\item смерть
\end{itemize}

