\question{Основные этапы, черты и проблемы средневековой философии.}

В Средние века происходит становление христианства. Поэтому на первом этапе происходит трансформация философских учений греческих философов, например, учение о Логосе в учение о христианском боге (Филон Александрийский). Начинает формироваться теологическая философия.

Схоластика -- господствующее направление в средневековой теологической философии, для которого характерны оторванность от реальной жизни, крайний догматизм, консерватизм, полное и беспрекословное подчинение духовным идеям, учительство. Цель -- философски обосновать религиозное учение (<<Философия -- служанка богословия>>).

Церковные догматы:
\begin{enumerate}
	\item догмат творения (сводится к сотворению мира богом, всемогуществу, первичности, вечности, вездесуществу, подлинности, постоянству бога);
	\item догмат откровения (мир можно познать, только познав бога, бога нельзя познать, он разрешил познать себя через библию путём толкования при наличии веры).
\end{enumerate}

В схоластике выделяют:
\begin{itemize}
	\item реалистов -- они утверждали, что подлинны (первичны) универсалии -- общие понятия вещей (по сути платоновские эйдосы), то есть сначала появилось понятие <<животные>> и только потом понятие <<заяц>> (Ансельм Кентерберийский, Гильям из Шампо);
	\item номиналистов -- они утверждали, что подлинны сами вещи (по сути аристотелевские индивидуумы), а понятия лишь имена вещей (<<универсалии существуют после вещей>>, то есть сначала появляются понятия <<заяц>>, <<волк>> и т. д. и только потом понятие <<животные>>) (Дунс Скот, Пьер Абеляр).
\end{itemize}

Наиболее яркими представителями средневековой теологической философии были Августин Блаженный (354 - 430) и Фома Аквинский (1225 - 1274).

Аврелий Августин Блаженный:
\begin{itemize}
	\item история -- борьба двух царств -- Земного (государство) и Божественного (Церковь);
	\item Земное царство однажды проиграет Божественному;
	\item Земное царство необходимо, чтобы карать грешников;
	\item терпите бедность и нищету, будьте праведниками и вы попадёте в рай;
	\item Августин также размышлял о боге, его существовании и творении.
\end{itemize}

Фома Аквинский:
\begin{itemize}
	\item пять доказательств существования бога:
	\begin{enumerate}
		\item \textit{движение}: всё, что движется, движется кем-то, что есть перводвигатель -- бог;
		\item \textit{причина}: всё, что существует, имеет причину, что есть первопричина -- бог;
		\item \textit{необходимое и случайное}: всё случайное зависит от необходимого, что есть первоначальная необходимость -- бог;
		\item \textit{степени качеств}: всё, что существует, имеет различные степени качеств (либо лучше, либо хуже и т. д.), что есть совершенство -- бог;
		\item \textit{цель}: всё, что существует, движется к цели, что придаёт смысл всему -- бог;
	\end{enumerate}
	\item разделил всё на сущность (эссенцию, т. е. божественные мысли о вещи) и экзистенцию (т. е. реальные вещи), бог превращает эссенцию в экзистенцию;
	\item идея вещи троична, существует в божественной мысли, в самой вещи, в памяти человека;
	\item разделил познание посредством веры и посредством разума, выделил круг вопросов познаваемых посредством разума (бессмертие души, существование бога и т. д.) и вопросов познаваемых через веру (сотворение мира, троичность бога и т. д.).
\end{itemize}