\input{../.preambles/00-lectures}
\input{../.preambles/10-russian}
\input{../.preambles/20-math}

\renewcommand{\arraystretch}{1.2}

\newcommand{\header}[1]{\vspace*{.3em}\emph{#1}\vspace*{.2em}}
\newcommand{\D}{\,\Delta}
\newcommand{\pnder}[3]{\frac{\partial^{#1} #2}{\partial #3^{#1}}}

\begin{document}
\emph{1. Понятие о ДУЧП, примеры. Методы решения ДУЧП. Типы ДУЧП.}

Уравнение, связывающее неизвестную функцию \( u(x_1, \ldots, x_n \), независимые
переменные \( x_1 \), \ldots, \( x_n \) и частные производные от неизвестной
функции, называется дифференциальным уравнением с частными производными. Оно
имеет вид:
\[
    F\left(x_1, x_2, \ldots, x_n, u, \pder{u}{x_1}, \pder{u}{x_2}, \ldots,
    \pder{u}{x_n}, \ppder{u}{x_1}, \pcder{u}{x_1}{x_2}, \ppder{u}{x_2}, \ldots,
    \pnder{n}{u}{x_m}\right) = 0,
\]
где \( F \) -- заданная функция своих аргументов.

Порядок старшой производной, входящей в уравнение, называется порядком ДУЧП.

\begin{table}[h!]
    \begin{tabular}{m{.48\textwidth}cm{.48\textwidth}}
        \header{Типы ДУЧП}

        Существует 6 способов классификации ДУЧП:
        \begin{enumerate}\itemsep-.4em
            \item порядок уравнения;
            \item число переменных;
            \item линейность (функция и все её производные входят в уравнение
            линейным образом);
            \item однородность;
            \item вид коэффициентов (постоянные либо переменные);
            \item три основных вида линейные ДУЧП II:
            \begin{itemize}\itemsep-.4em
                \item параболические;
                \item эллиптические;
                \item гиперболические.
            \end{itemize}

            Также при переменных коэффициентах возможен <<смешанный>> тип
            уравнения.
        \end{enumerate}
        % -----------------------
        & \hfill &
        % -----------------------
        \header{Методы решения ДУЧП}

        \begin{enumerate}
            \item Метод разделения переменных (метод Фурье);
            \item метод интегральных преобразований;
            \item метод преобразования координат;
            \item метод преобразования зависимой переменной;
            \item численные методы;
            \item методы теории возмущений;
            \item метод функций Грина;
            \item метод интегральных уравнений;
            \item вариационные методы;
            \item метод разложения по собственным функциям.
        \end{enumerate}
    \end{tabular}
\end{table}

\header{Примеры}

\begin{enumerate}
    \item Волновое уравнение:
    \[
        \ppder{u}{t} = c^2\left(\ppder{u}{x} + \ppder{u}{y} +
        \ppder{u}{z}\right);
    \]

    \item уравнение теплопроводности:
    \[
        \pder{u}{t} = \alpha^2\left(\ppder{u}{x} + \ppder{u}{y} +
        \ppder{u}{z}\right);
    \]

    \item уравнение Лапласа:
    \[
        \ppder{u}{x} + \ppder{u}{y} + \ppder{u}{z} = 0.
    \]
\end{enumerate}

\newpage % ---------------------------------------------------------------------

\emph{2. Математическая модель теплопроводности. Некоторые уравнения
диффузионного типа.}

\header{Математическая модель теплопроводности} представляет собой систему четырех
уравнений:
\[
    \left. \begin{array}{rl}
        \text{ДУЧП:} & \displaystyle \pder{u}{t} = \alpha^2\ppder{u}{x}; 
        \vspace*{.4em} \\
        \text{ГУ:} & \left\{ \begin{array}{l}
            u(0, t) = f_1(t), \\
            u(l, t) = f_2(t); 
        \end{array} \right. \\
        \text{НУ:} & u(x, 0) = \phi(x).
    \end{array} \right\}
\]

\header{Некоторые уравнения диффузионного типа:}
\begin{enumerate}
    \item теплообмен через боковую поверхность пропорционально разности
    температур:
    \[
        \pder{u}{t} = \alpha^2\ppder{u}{x} - \beta(u - u_0);
    \]
    
    \item внутренний источник тепла:
    \[
        \pder{u}{t} = \alpha^2\ppder{u}{x} + f(x, t);
    \]
    
    \item уравнение конвективной диффузии:
    \[
        \pder{u}{t} = \alpha^2 \ppder{u}{x} - v\pder{u}{x}.
    \]
\end{enumerate}

\newpage % ---------------------------------------------------------------------

\emph{3. Граничные условия в задачах диффузионного типа.}

\newpage % ---------------------------------------------------------------------

\emph{4. Вывод уравнения теплопроводности.}

\newpage % ---------------------------------------------------------------------

\emph{5. Метод разделения переменных и его применение в диффузионных задачах.}

\newpage % ---------------------------------------------------------------------

\emph{6. Преобразование неоднородных ГУ в однородные. Преобразование зависящих
от времени ГУ в нулевые.}

\newpage % ---------------------------------------------------------------------

\emph{7. Задача теплопроводности с производной в ГУ. Задача Штурма-Лиувилля и её
свойства.}

\newpage % ---------------------------------------------------------------------

\emph{8. Преобразование задачи с теплообменом через боковую поверхность к задаче
с теплоизолированной боковой поверхностью.}

\newpage % ---------------------------------------------------------------------

\emph{9. Решение неоднородных ДУЧП методом разложения по собственным функциям.}

\newpage % ---------------------------------------------------------------------

\emph{10. Интегральные преобразования. Спектр функции. Синус- и
косинус-преобразования производных.}

\newpage % ---------------------------------------------------------------------

\emph{11. Решение диффузионной задачи на полупрямой методом
синус-преобразования. Интерпретация решения.}

\newpage % ---------------------------------------------------------------------

\emph{12. Ряд Фурье и его коэффициенты. Теорема Дирихле. Дискретный частотный
спектр периодической функции. Преобразование Фурье.}

\newpage % ---------------------------------------------------------------------

\emph{13. Свойства преобразования Фурье. Решение задачи о распространении тепла
в бесконечном стержне с заданной начальной температурой.}

\newpage % ---------------------------------------------------------------------

\emph{14. Преобразование Лапласа и его свойства. Решение задачи о
теплопроводности в полубесконечной среде.}

\newpage % ---------------------------------------------------------------------

\emph{15. Принцип Дюамеля.}

\newpage % ---------------------------------------------------------------------

\emph{16. Решение задачи конвективного переноса методом преобразования Лапласа.}

\newpage % ---------------------------------------------------------------------

\emph{17. Уравнение колебаний струны и его интуитивная интерпретация.
Замечания.}

\newpage % ---------------------------------------------------------------------

\emph{18. Решение одномерного волнового уравнения с помощью формулы Даламбера.
Примеры применения формулы Даламбера в некоторых конкретных задачах.}

\newpage % ---------------------------------------------------------------------

\emph{19. Пространственно-временная интерпретация формулы Даламбера. Решение
задачи для полубесконечной струны.}

\newpage % ---------------------------------------------------------------------

\emph{20. Волновое уравнение и три типа граничных условий.}

\newpage % ---------------------------------------------------------------------

\emph{21. Решение задачи о колебаниях ограниченной струны методом разделения
переменных.}

\newpage % ---------------------------------------------------------------------

\emph{22. Колебания балки (ДУЧП IV).}

\newpage % ---------------------------------------------------------------------

\emph{23. Переход к безразмерным переменным на примере диффузионной задачи.
Пример преобразования гиперболической задачи к безразмерному виду.}

\newpage % ---------------------------------------------------------------------

\emph{24. Классификация ДУЧП II. Приведение к каноническому виду.}

\newpage % ---------------------------------------------------------------------

\emph{25. Волновое уравнение в свободном пространстве (2- и 3-мерные задачи).}

\newpage % ---------------------------------------------------------------------

\emph{26. Конечные синус- и косинус-преобразования Фурье. Решение краевой задачи
с неоднородным волновым уравнением.}

\newpage % ---------------------------------------------------------------------

\emph{27. Принцип суперпозиции. Разложение смешанной задачи на две более
простые. Разделение переменных и интегральные преобразования как проявление
принципа суперпозиции.}

\newpage % ---------------------------------------------------------------------

\emph{28. Уравнения первого порядка (метод характеристик). Общая стратегия
решения ДУЧП I.}

\newpage % ---------------------------------------------------------------------

\emph{29. Нелинейные ДУЧП I. Вывод и применение закона сохранения к задаче о
дорожном движении.}

\newpage % ---------------------------------------------------------------------

\emph{30. Системы ДУЧП. Решение простейшей линейной системы.}

\newpage % ---------------------------------------------------------------------

\emph{31. Колебания мембраны. Решение задачи на собственные значения для
уравнения Гельмгольца.}

\end{document}
