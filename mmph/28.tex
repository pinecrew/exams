\chapter{Уравнения первого порядка (метод характеристик). Общая стратегия
решения ДУЧП I.}

Рассмотрим линейное уравнение
\[
    a(x, t)\pder{u}{x} + b(x, t)\pder{u}{t} + c(x, t)u = 0, \quad
    u(x, 0) = \phi(x).
\]

Перейдём к новым переменным \( (s, \tau) \) так, чтобы
\[
    \left\{ \begin{array}{l}
        \ds \der{x}{s} = a(x, t), \\[.4em]
        \ds \der{t}{s} = b(x, t).
    \end{array} \right.
\]

В этом случае уравнение преобразуется к виду
\[
    \der{u}{s} + c(x, t)u = 0.
\]

Переменную \( \tau \) введём следующим образом: \( \ds \tau = x\bigr|_{s=0} \).

При этом начальное условие преобразуется к виду \( u(0) = \phi(\tau) \), а вся
задача преобразуется в следующую задачу Коши:
\[
    \left\{ \begin{array}{l}
        \ds \der{u}{s} + cu = 0, \\
        u(0) = \phi(\tau),
    \end{array} \right.
\]
которая элементарно решается. Возвращаясь к координатам \( (x, t) \) получаем
решение исходной задачи.
\newpage
