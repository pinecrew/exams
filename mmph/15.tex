\chapter{Принцип Дюамеля.}

\begin{minipage}{.55\textwidth}
    Рассмотрим задачу:

    Попробуем решить её при помощи преобразования Лапласа по времени:
\end{minipage}
\hfill
\begin{minipage}{.4\textwidth}
    \[
        \left\{ \begin{array}{l}
            \ds \pder{u}{t} = \alpha^2 \ppder{u}{x}; \\
                \left\{ \begin{array}{l}
                    u(0, t) = 0, \\
                    u(l, t) = f(t); 
                \end{array} \right. \\
            u(x, 0) = 0.
        \end{array} \right.
    \]
\end{minipage}

\[
    \left\{ \begin{array}{l}
        \dder{\bar{u}}{x} - \frac{s}{\alpha^2}\bar{u} = 0, \\
        \bar{u}(0) = 0, \\
        \bar{u}(l) = F(s);
    \end{array} \right.
    \Rightarrow
    \bar{u}(x, s) = F(s)\left[\frac{\sh\left(\frac{x}{\alpha}\sqrt{s}\right)}
    {\sh\left(\frac{l}{\alpha}\sqrt{s}\right)}\right].
\]

Заметим, что если \( f(t) = A = \const \), то соответствующее ему решение имеет
вид
\[
    w(x, t) = \frac{A}{l}x + \frac{2}{\pi}\sum\limits_{n=1}^\infty
    \frac{(-1)^n}{n}\e^{-\left(\alpha\frac{\pi n}{l}\right)^2t}\cdot
    \sin\frac{\pi n}{l}x,
\]
а Лаплас-образ: \( \ds \bar{w} = \frac{1}{s}\left[\frac{\sh\left(
\frac{x}{\alpha}\sqrt{s}\right)}{\sh\left(\frac{l}{\alpha}\sqrt{s}\right)}
\right] \).
Тогда, представляя \( \bar{u}(x, s) \) в виде \( sF(s)\cdot\bar{w}(s) \) и
учитывая соотношение \( \ds L\left[\pder{w}{t}\right] = s\bar{w} - w(x, 0) \),
имеем:
\begin{align*}
    \bar{u}(x, s) = L[f]\cdot L\left[\pder{w}{t}\right] \Rightarrow
    u(x, t) = f(t)*\pder{w}{t} = \int\limits_0^t \pder{w}{t}(x, t-\tau)\cdot
    f(\tau)\d\tau = \\
    = \int\limits_0^t w(x, t - \tau)f'(\tau)\d\tau + f(0)w(x, t).
\end{align*}

Также, если известен отклик на температурный импульс \( \delta(t) \), то
соответствующая формула имеет вид:
\[
    u(x, t) = \int\limits_0^t v(x, t - \tau)f(\tau)\d\tau,
\]
где \( v(x, t) \) -- отклик системы на температурный импульс.

\newpage
