\chapter{Преобразование задачи с теплообменом через боковую поверхность к задаче
с теплоизолированной боковой поверхностью.}

Задача с теплообменом через боковую поверхность имеет вид:
\[
    \left. \begin{array}{rl}
        \text{ДУЧП:} & \ds \pder{u}{t} = \alpha^2\ppder{u}{x} - \beta u;
        \vspace*{.4em} \\
        \text{ГУ:} & \left\{ \begin{array}{l}
            u(0, t) = 0, \\
            u(l, t) = 0; 
        \end{array} \right. \\
        \text{НУ:} & u(x, 0) = \phi(x),
    \end{array} \right\}
\]

Эта задача может быть сведена к задаче с теплоизолированной боковой поверхностью
путем замены \( u(x, t) = \e^{-\beta t}\cdot\bar{u}(x, t) \). Действительно,
\begin{align*}
    \pder{u}{t} = -\beta\cdot\e^{-\beta t}\cdot\bar{u}(x, t) +
    \e^{-\beta t}\pder{\bar{u}}{t}, \\
    \ppder{u}{x} = \e^{-\beta t}\ppder{\bar{u}}{x}.
\end{align*}

Подставляя в исходную задачу, имеем:
\[
    \left. \begin{array}{rl}
        \text{ДУЧП:} & \ds \pder{\bar{u}}{t} = \alpha^2\ppder{\bar{u}}{x};
        \vspace*{.4em} \\
        \text{ГУ:} & \left\{ \begin{array}{l}
            \bar{u}(0, t) = 0, \\
            \bar{u}(l, t) = 0; 
        \end{array} \right. \\
        \text{НУ:} & \bar{u}(x, 0) = \bar{\phi}(x),
    \end{array} \right\}
\]

\newpage
