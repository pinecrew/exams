\chapter{Преобразование Лапласа и его свойства. Решение задачи о
теплопроводности в полубесконечной среде.}

\section{Преобразование Лапласа}
Интегральное преобразование \( \ds L[f] = \int\lni f(t)\e^{-st}\d t \)
называется преобразованием Лапласа.

Достаточное условие существования преобразования Лапласа: при любом \( A > 0 \)
функция \( f(t) \) кусочно-непрерывна на \( [0, A] \); существуют постоянные
\( M \), \( a \), \( T \) такие, что \( \ds |f(t)| \le M\e^{at} \) для всех
\( t > T \). В этом случае преобразование Лапласа существует при всех
\( s > a \).

Свойства преобразования Лапласа:
\begin{enumerate}
    \item Обратимость: \( \ds L^{-1}\bigl[L[f]\bigr] = f \),
    
    где обратное преобразование: \( \ds L^{-1}[F] = \frac{1}{2\pi\i}
    \int\limits_{c-\i\infty}^{c+\i\infty} F(s)\e^{st}\d s \);
    
    \item преобразование частных производных:\\
    \(
        \ds L_t\left[\pnder{n}{u}{t}\right] = s^nL_t[u] - 
        \sum\limits_{k=0}^{n-1} s^k u^{(n - 1 - k)},\quad 
        \ds L_t\left[\pnder{n}{u}{x}\right] = \pnder{n}{}{x}L_t[u];
    \)
    
    \item свёртка: \( L[f*g] = L[f]\cdot L[g] \),
    \( \ds f*g = \int\limits_0^t f(t-\tau)g(\tau)\d\tau \);
    
    \item линейность: \( L[af + bg] = aL[f] + bL[g] \).
\end{enumerate}

\section{Задача о теплопроводности в полубесконечной среде}
\begin{minipage}{.6\textwidth}
    Рассмотрим теперь задачу о теплопроводности в полубесконечной среде.

    Сделаем преобразование Лапласа по переменной \( t \): \( u \to \bar{u} \).
        
    Решая, получим: \( \ds \bar{u} = C_1\e^{\frac{\sqrt{s}}{\alpha}x} +
    C_2\e^{-\frac{\sqrt{s}}{\alpha}x} + \frac{u_0}{s} \).

    Подставляя в ГУ, находим: \( C_1 = 0 \), \( \ds C_2 =
    -\frac{u_0}{s\left(1 + \frac{\sqrt{s}}{\alpha h}\right)} \), тогда:
    \[
        \bar{u}(x) = u_0\left(\frac{1}{s} - \frac{\e^{-\frac{\sqrt{s}}
        {\alpha}x}}{s\left(1 + \frac{\sqrt{s}}{\alpha h}\right)}\right).
    \]
\end{minipage}
\hfill
\begin{minipage}{.3\textwidth}
    \flushright
    \[
        \left\{ \begin{array}{l}
            \ds \pder{u}{t} = \alpha^2 \ppder{u}{x}; \\[.5em]
            \ds \pder{u}{x}(0,t) 0 - hu(0, t) = 0; \\
            u(x, 0) = u_0.
        \end{array} \right.
    \]
    Преобразованная задача:
    \[
        \left\{ \begin{array}{l}
            \ds s\bar{u}(x) - u_0 = \alpha^2\dder{\bar{u}}{x}; \\
            \ds \der{\bar{u}}{x}(0) = h\bar{u}(0).
        \end{array} \right.
    \]
\end{minipage}

Сделав обратное преобразование, получим:
\[
    u(x, t) = u_0 - u_0\left[\erfc\left(\frac{x}{2\alpha\sqrt{t}}\right) + 
    \erfc\left(\frac{x}{2\alpha\sqrt{t}} + \alpha h\sqrt{t}\right)\cdot
    \e^{xh + \alpha^2h^2 t}\right].
\]

\newpage
