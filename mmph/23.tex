\chapter{Переход к безразмерным переменным на примере диффузионной задачи.
Пример преобразования гиперболической задачи к безразмерному виду.}

Рассмотрим смешанную \emph{диффузионную задачу} для стержня:

\begin{minipage}{.35\textwidth}
\[
    \left. \begin{array}{rl}
        \text{ДУЧП:} & \ds \pder{u}{t} = \alpha^2\ppder{u}{x}, \\[.4em]
        \text{ГУ:} & \left\{ \begin{array}{l}
            u(0, t) = T_1, \\
            u(l, t) = T_2,
        \end{array} \right. \\
        \text{НУ:} & \ds u(x, 0) = T_0\sin\frac{\pi x}{l}.
    \end{array} \right\}
\]
\end{minipage}
\hfill
\begin{minipage}{.58\textwidth}
    Сделаем переход к безразмерным величинам:
    \[
        u \to \u, \qquad x \to \xi, \qquad t \to \tau.
    \]
    
    Зависимую переменную будем преобразовывать исходя из граничных условий.
    Потребуем, чтобы левое обратилось в 0, а правое в 1:
    \( \u(0, t) = 0 \), \( \u(l, t) = 1 \).
\end{minipage}

\begin{minipage}{.57\textwidth}
    Это будет выполняться, если \( \u(x, t) = \cfrac{u(x, t) - T_1}{T_2 - T_1} \).
    Таким образом, задача преобразуется к виду:
\end{minipage}
\hfill
\begin{minipage}{.3\textwidth}
\[
    \left\{ \begin{array}{l}
        \ds \pder{\u}{t} = \alpha^2\ppder{\u}{x}, \\[.4em]
        \left\{ \begin{array}{l}
            \u(0, t) = 0, \\
            \u(l, t) = 1,
        \end{array} \right. \\
        \ds \u(x, 0) = \frac{T_0\sin\frac{\pi x}{l}- T_1}{T_2 - T_1}.
    \end{array} \right.
\]
\end{minipage}

\begin{minipage}{.6\textwidth}
    Теперь преобразуем переменную \( x \). Имеет смысл привести длину отрезка,
    на котором решается задача, к 1, то есть \( \xi~=~x/l \). Теперь задача
    может быть записана в виде:
\end{minipage}
\hfill
\begin{minipage}{.3\textwidth}
\[
    \left\{ \begin{array}{l}
        \ds \pder{\u}{t} = \frac{\alpha^2}{l^2}\ppder{\u}{\xi}, \\[.4em]
        \left\{ \begin{array}{l}
            \u(0, t) = 0, \\
            \u(1, t) = 1,
        \end{array} \right. \\
        \ds \u(\xi, 0) = \frac{T_0\sin\pi\xi - T_1}{T_2 - T_1}.
    \end{array} \right.
\]
\end{minipage}

\begin{minipage}{.6\textwidth}
Наконец, преобразуем \( t \to \tau \) так, чтобы обезразмерить запись ДУЧП, то
есть
\[
    \pder{\u}{\tau} = \ppder{\u}{\xi}, \text{ отсюда }
    \tau = \frac{\alpha^2}{l^2}t.
\]
В итоге, имеем безразмерный вид диффузионной задачи,
где
\[
    \u = \frac{u(x, t) - T_1}{T_2 - T_1}, \quad \xi = x/l, \quad
    \tau = \frac{\alpha^2}{l^2}t.
\]
\end{minipage}
\hfill
\begin{minipage}{.3\textwidth}
\[
    \left\{ \begin{array}{l}
        \ds \pder{\u}{\tau} = \ppder{\u}{\xi}, \\[.4em]
        \left\{ \begin{array}{l}
            \u(0, \tau) = 0, \\
            \u(1, \tau) = 1,
        \end{array} \right. \\
        \ds \u(\xi, 0) = \frac{T_0\sin\pi\xi - T_1}{T_2 - T_1}.
    \end{array} \right.
\]
\end{minipage}

\newpage
А сейчас рассмотрим пример преобразования \emph{гиперболической задачи} к
безразмерному виду.

\begin{minipage}{.4\textwidth}
\[
    \left. \begin{array}{l}
        \ds \ppder{u}{t} = c^2\ppder{u}{x}, \\[.4em]
        u(0, t) = u(l, t) = 0, \\
        \left\{ \begin{array}{l}
            \ds u(x, 0) = \sin\frac{\pi x}{l} + \frac{1}{2}
            \sin\frac{3\pi x}{l}, \\[.4em]
            \ds \pder{u}{t}(x, 0) = 0.
        \end{array} \right.
    \end{array} \right\}
\]
\end{minipage}
\hfill
\begin{minipage}{.53\textwidth}
    Как видно, это задача о струне с закрепленными концами. Так как граничные
    условия уже безразмерные, то нужно преобразовать только \( x \) и \( t \).
    
    Эти преобразования имеют вид:
    \[
        \xi = x/l \text{ и } \tau = ct/l.
    \]
\end{minipage}

\vspace*{1em}
Преобразованная задача имеет вид:

\begin{minipage}{.4\textwidth}
\[
    \left. \begin{array}{l}
        \ds \ppder{u}{\tau} = \ppder{u}{\xi}, \\[.4em]
        u(0, \tau) = u(1, \tau) = 0, \\
        \left\{ \begin{array}{l}
            \ds u(\xi, 0) = \sin\pi\xi + \frac{1}{2}\sin\frac3\pi\xi, \\[.4em]
            \ds \pder{u}{\tau}(\xi, 0) = 0.
        \end{array} \right.
    \end{array} \right\}
\]
\end{minipage}
\hfill
\begin{minipage}{.55\textwidth}
    \vspace*{.7em}
    Её решение нетрудно найти:
    \[
        u(\xi, \tau) = \sin\pi\xi\cos\pi\tau +
        \frac{1}{2}\sin3\pi\xi\cos3\pi\tau.
    \]
    Возвращаясь к исходной задаче, получаем ответ:
    \[
        u(x, t) = \sin\frac{\pi x}{l}\cos\frac{\pi ct}{l} +
        \frac{1}{2}\sin\frac{3\pi x}{l}\cos\frac{3\pi ct}{l}.
    \]
\end{minipage}
\newpage
