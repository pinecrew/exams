\question{Газовый разряд. Виды и характеристики газового разряда. Рождение и
  исчезновение носителей зарядов}

В обыкновенных условиях, то есть при отсутствии внешнего воздействия, газ
является диэлектриком. Ток будет течь только при наличии внешнего ионизатора,
например, космического излучения. Если ионизатор убрать, то ток исчезнет. Такой
разряд называется \emph{несамостоятельным} (рис.~\pic{32VAC}-I).
\begin{figure}[h!]
  \center
  \includegraphics[width=.3\textwidth]{32_VAC} \hspace{1em}
  \includegraphics[width=.3\textwidth]{32_Townsend} \hspace{1em}
  \includegraphics[width=.3\textwidth]{32_U_light} \\
  \parbox{.3\textwidth}{\caption{ВАХ газового разряда} \label{pic32VAC}}
    \hspace{1em}
  \parbox{.3\textwidth}{\caption{К описанию эффекта Таунсенда}
    \label{pic32Townsend}} \hspace{1em}
  \parbox{.3\textwidth}{\caption{Зависимость \( U_\text{з}(p \cdot d\))}
    \label{pic32U}} \\
  \parbox{.37\textwidth}{\footnotesize I~-- несамостоятельный разряд,
    II~-- самостоятельный разряд; 1~-- переходная область, 2~-- тлеющий
    разряд, 3~-- дуговой разряд}
  \parbox{.62\textwidth}{\ }
\end{figure}

При повышении разности потенциалов между анодом и катодом энергия электронов
будет увеличиваться, и при достижении некоторого напряжения этой энергии будет
достаточно, чтобы при столкновении электронов с нейтральными атомами произвести
ионизацию. Ток увеличивается, это явление называется газовым усилением. Однако
если убрать внешний ионизатор, ток упадет до нуля, следовательно, такой разряд
также является несамостоятельным.

При дальнейшем повышении напряжения, положительные ионы, попадая на катод,
начинают вызывать вторичную эмиссию. При этом, даже если убрать внешний
ионизатор, ток не прекратится. Такой разряд называется \emph{самостоятельным}
(рис.~\pic{32VAC}-II). Область~2 на ВАХ называется \emph{тлеющим разрядом}.

При дальнейшем увеличении напряжения ионы будут разогревать катод, проявится
явление термоэлектронной эмиссии (область~3). Такой разряд называется
\emph{дуговым}.

\subquestion{Эффект Таунсенда}

Эффектом Таунсенда называется переход разряда из несамостоятельного в
самостоятельный. Пусть \( \alpha \)~-- коэффициент объемной ионизации, равный
количеству электронно-ионных пар, образуемых одним электроном на единице пути.

Коэффициент \( \alpha \) зависит от давления \( p \) и напряженности
электрического поля \( E \).

Выделим в газовом промежутке слой толщиной \( dx \) (рис.~\pic{32Townsend}).
Один электрон, пролетая этот слой, создаст \( \alpha\,dx \) электронно-ионных
пар. Если плоскость \( x = \const \) пересекает ток электронов \( I_e \), то в
слое \( dx \) он возрастет на величину
\begin{equation}
  dI_e = \alpha I_e\,dx.
  \label{eq32dIe}
\end{equation}

Если положить, что \( U = \const \), то \( \alpha = \const(x) \). Проинтегрируем
\eqref{eq32dIe}:
\[
  I_e = I_e(0)e^{\alpha x},
\]
где \( I_e(0) \)~-- прикатодный ток (ток, создаваемый внешним ионизатором).

Чтобы разряд не прекращался, необходимо поддерживание тока \( I_e(0) \) самим
разрядом, то есть возникал бы поток положительно заряженных ионов, движущихся к
катоду.

Данный ток легко определить из закона сохранения заряда:
\[
  I_i(0) = I_e(d) - I_e(0); \qquad I_i(0) = I_e(0) \Big[ e^{\alpha d} - 1 \Big].
\]

Пусть каждый пришедший на катод ион выбивает в среднем \( \gamma(p, E) \)
вторичных электронов. Следовательно, из катода начнет эмитироваться ток
вторичных электронов \( I_2 \):
\[
  I_2 = \gamma I_i(0) = \gamma I_e(0) \Big[ e^{\alpha d} - 1 \Big].
\]

Таким образом, полный ток, эмитируемый с катода:
\[
  I_e(0) = I_1 + I_2,
\]
где \( I_1 \)~-- ток, вызываемый внешним ионизатором.

\[
  I_e(0) = I_1 + \gamma I_e(0) \Big[ e^{\alpha d} - 1 \Big]; \quad
    I_e(0) = \frac{I_1}{1 - \gamma(e^{\alpha d} - 1)}; \quad
    I_e(d) = \frac{I_1 e^{\alpha d}}{1 - \gamma(e^{\alpha d} - 1)}.
\]

Если убрать внешний ионизатор, то \( I_1 \to 0 \) и \( I_e(d) \to 0 \). Но
можно подобрать \( \gamma \) и \( \alpha \) таким образом, чтобы
\(
  \gamma \Big[ e^{\alpha d} - 1 \Big] \to 1
\).

Величину \( \mu = \gamma \Big[ e^{\alpha d} - 1 \Big] \) называют коэффициентом
воспроизводства. \( \mu = 1 \) является критерием Таунсенда~-- критерием
перехода из несамостоятельного в самостоятельный.

Так как давление \( p \) во многих случаях остается постоянным, то
\( \gamma = \gamma(U) \), \( \alpha = \alpha(U) \). Потенциал, при котором
происходит переход разряда из несамостоятельного в самостоятельный, называется
потенциалом зажигания \( U_\text{з} = U_\text{з}(p \cdot d) \) (рис.~\pic{32U}).

\subquestion{Процессы потери электронов}

\hspace{1em}\textbf{Диффузия}

Закон Фика гласит: \( \vec{M} = -D\,\gradient\rho \), где \( M \)~-- масса
вещества, диффундирующего в единицу времени через единичную площадку,
\( \rho \)~-- плотность вещества.

Домножим на \( e \) и разделим на \( m_e \):
\begin{equation}
  \vec{j} = D^* \gradient n,
  \label{eq32diff}
\end{equation}
где \( n \)~-- объемная плотность заряда, \( j \)~-- плотность тока.

С другой стороны, есть уравнение непрерывности:
\begin{equation}
  \der{n}{t} - \divergence\vec{j} - Q = 0,
  \label{eq32nobreak}
\end{equation}
где \( Q = nq_i \)~-- скорость образования или потери электронов в рассматриваемом
объеме, \( q_i \)~-- скорость образования или потери электронов в рассматриваемом
объеме, отнесенная к одному электрону.

Подставим \eqref{eq32diff} в \eqref{eq32nobreak}
\[
  \pder{n}{t} = D^*\D n + nq_i; \qquad
    \frac{1}{n}\pder{n}{t} - q_i = \frac{D^*}{n}\D n.
\]

Поскольку в левой части стоит производная по времени, а с правой~-- по
координатам, то, чтобы решение существовало при любых координатах и времени,
каждая из частей должна быть равна некоторой константе. Обозначим ее
\( -\gamma \). Получаем систему уравнений:
\[
  \left\{
    \begin{array}{l}
      \displaystyle\pder{n}{t} = (q_i - \gamma)n; \\[.6em]
      \D n + \dfrac{\gamma}{D}n = 0.
    \end{array}
  \right.
\]

Проинтегрируем первое уравнение системы, получим
\(
  n(t) = n_0 e^{(q_i - \gamma)t}
\).

Для того, чтобы проинтегрировать второе уравнение, сделаем ряд допущений:
\begin{itemize}
  \item допустим, что рассматриваемая область является параллелепипедом
    размерами \( a\times b\times c \);
  \item допустим, что на границах \( n = 0 \).
\end{itemize}

Тогда
\[
  n(x, y, z) = n_1\sin\frac{\pi x}{a} \sin\frac{\pi y}{b} \sin\frac{\pi z}{c};
    \quad \frac{\gamma}{D^*} = \frac{\pi^2}{a^2} + \frac{\pi^2}{b^2} +
    \frac{\pi^2}{c^2}.
\]

Если рассматриваемая система является плоским диодом, то \( b\to\infty \) и
\( c\to\infty \), и тогда
\[
  \gamma = \frac{\pi^2}{a^2}D^*; \quad
    \frac{\gamma}{D^*} = \frac{1}{\Lambda^2}; \quad
    n(t) = n_0 e^{\left( q_i - \frac{\pi^2}{a^2}D^* \right)t} =
    n(t) = n_0 e^{\left( q_i - \frac{D^*}{\Lambda^2} \right)t},
\]
где \( \Lambda \)~-- характерная диффузионная длина.

\hspace{1em}\textbf{Прилипание электронов к нейтральным атомам}

Из-за того, что масса атомов газа много больше массы электрона, то система
обладает малой подвижностью. Поэтому, хоть атом с <<прилипшим>> к нему
электроном и имеет отрицательный заряд, он будет двигаться к аноду с очень малой
скоростью. Из-за этого этот процесс эквивалентен потере электронов.

Введем коэффициент \( \alpha_\text{п} \)~-- число прилипаний на единицу длины.
Тогда
\[
  dn = -\alpha_\text{п}n\,dx; \qquad n = n(0)e^{-\alpha_\text{п}x}.
\]

\hspace{1em}\textbf{Рекомбинация}

Рекомбинация~-- процесс потери электронов при их столкновении с положительными
ионами.

Когда происходит рекомбинация, плотность положительных ионов и электронов
изменяется, а скорость этого изменения пропорциональна количеству частиц каждого
рода, так что
\[
  \der{n_+}{t} = \der{n_e}{t} = -\alpha_r n_+ n_e,
\]
где \( \alpha_r \)~-- коэффициент рекомбинации. Обычно \( n_+ = n_e = n \),
тогда
\[
  \der{n}{t} = -\alpha_r n^2; \qquad
    \frac{1}{n} = \alpha_r t + \frac{1}{n_0},
\]
где \( n_0 \)~-- начальная плотность частиц.

Отсюда
\(
  \alpha_r = \dfrac{1}{t} \left( \dfrac{1}{n} - \dfrac{1}{n_0} \right)
\).
