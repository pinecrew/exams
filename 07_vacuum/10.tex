\question{Аксиально-симметричные магнитные поля. Распределение поля в
  параксиальной области}

Для определения характера зависимости траектории частицы в неоднородном
статическом магнитном поле необходимо знать пространственное распределение
\( \vec{B}_0(z) \). Примем, что поле \( \vec{B}_0 \) не зависит от угла поворота
вдоль оси симметрии \( z \), т.~е. \( \vec{B}_0 = \vec{B}_0(r, z) \).

Представим вектор индукции магнитного поля через векторный потенциал\( \vec{A} \):
\[
  \vec{B}_0 = \rotor\vec{A},
\]
или, расписывая в скалярной форме, получим
\begin{equation}
  \left\{
    \begin{array}{l}
      \displaystyle B_{0z} = \frac{1}{r}\pder{}{r} \Big( rA_\phi \Big); \\
      \displaystyle B_{0r} = -\pder{A_\phi}{z}; \\
      \displaystyle B_{0\phi} = \pder{A_r}{z} - \pder{A_z}{r}.
    \end{array}
  \right.
  \label{eq10B0rotor}
\end{equation}

С другой стороны, из уравнений Максвелла при отсутствии сторонних токов для
статического случая \( \rotor\vec{B}_0 = 0 \). Отсюда:
\begin{equation}
  \left\{
    \begin{array}{l}
      \displaystyle \Big( \rotor\vec{B}_0 \Big)_z = \frac{1}{r} \pder{}{r}
        \Big( rB_{0\phi} \Big) = 0; \\
      \displaystyle \Big( \rotor\vec{B}_0 \Big)_r = -\pder{B_{0\phi}}{z} = 0; \\
      \displaystyle \Big( \rotor\vec{B}_0 \Big)_\phi = \pder{B_{0r}}{z} -
        \pder{B_{0z}}{r} = 0.
    \end{array}
  \right.
  \label{eq10B0Maxwell}
\end{equation}

Из первый двух уравнений системы \eqref{eq10B0Maxwell} следует, что составляющая
\( B_{0\phi} \) не зависит от \( r \) и \( \phi \). Ее можно положить равной
нулю. Из \eqref{eq10B0rotor} и \eqref{eq10B0Maxwell} вытекают следующие
соотношения:
\begin{equation}
  B_{0z} = \frac{1}{r} \pder{}{r} \Big( rA_\phi \Big); \quad
    B_{0r} = -\pder{A_\phi}{z}; \quad
    \pder{B_{0r}}{z} - \pder{B_{0z}}{r} = 0.
  \label{eq10B0z}
\end{equation}

Подставляя в последнее из уравнений первые два получим дифференциальное
уравнение второго порядка:
\[
  \ppder{A_\phi}{z} + \ppder{A_\phi}{r} + \frac{1}{r}\pder{A_\phi}{r} -
    \frac{A_\phi}{r^2} = 0.
\]

Представим решение этого уравнения в виде ряда
\[
  A_\phi(r, z) = f_0(z) + r f_1(z) + r^2 f_2(z) + r^3 f_3(z) + \ldots
\]

Подставим в дифференциальное уравнение, учитывая, что при одинаковых степенях
\( r \) коэффициенты должны быть равны нулю. Сведем коэффициенты при различных
степенях в таблицу~\ref{tab10nums}
\begin{table}[h!]
  \center
  \caption{Коэффициенты при степенях \( r \)}
  \label{tab10nums}
  \begin{tabular}{|>{\( \displaystyle}C{.1}<{\)}|*{9}{>{\(}C{.05}<{\)}|}} \hline
    & -2 & -1 & 0 & 1 & 2 & 3 & 4 & 5 & \ldots \\ \hline
    -\frac{A_\phi}{r^2} & f_0 & f_1 & f_2 & f_3 & f_4 & f_5 & f_6 & f_7 &
      \ldots \rule{0pt}{2.2em} \\ \hline
    \frac{1}{r}\pder{A_\phi}{r} & 0 & f_1 & 2f_2 & 3f_3 & 4f_4 & 5f_5 &
      6f_6 & 7f_7 & \ldots \rule{0pt}{2.2em} \\ \hline
    \ppder{A_\phi}{r} & 0 & 0 & 2f_2 & 6f_3 & 12f_4 & 20f_5 & 30f_6 & 42f_7 &
      \ldots \rule{0pt}{2.2em} \\ \hline
    \ppder{A_\phi}{z} & 0 & 0 & f_0'' & f_1'' & f_2'' & f_3'' & f_4'' & f_5'' &
      \ldots \rule{0pt}{2.2em} \\ \hline
  \end{tabular}
\end{table}

Получаем \( f_0 = 0 \), следовательно, \( 3f_2 = -f_0'' = 0 \),
\( 15f_4 = -f_2'' = 0 \), \ldots, \( f_{2n} = 0 \); \\
\(
  f_3 = -f_1'' / (2 + 2 \cdot 3)
\),
\(
  f_5 = f_1^{(IV)} / \big[ (2 + 2 \cdot 3) \cdot (4 + 4 \cdot 5) \big]
\),
\ldots,
\(
  f_{2n + 1} = (-1)^n f_1^{(2n)} / \big[ 2^n 2^n n! (n+1)! \big]
\).

Таким образом,
\[
  A_\phi(r, z) = \sum_{n = 0}^\infty (-1)^n
    \frac{f_1^{(2n)}}{2^{2n} n! (n + 1)!} r^{2n + 1}.
\]

Для определения смысла функции \( f_1(z) \) подставим \( A_\phi(r, z) \) в
первое уравнение \eqref{eq10B0z}:
\[
  B_{0z} = \frac{A_\phi}{r} + \pder{A_\phi}{r} = 2f_1 - \frac{1}{2}r^2 f_1''(z)
    + \ldots
\]

При \( r = 0 \) \( f_1 = B_0 / 2 \), где \( B_0 \)~-- \( z \)-составляющая
магнитного поля на оси симметрии системы.

Подставляя в \( A_\phi(r, z) \), имеем
\[
  A_\phi(r, z) = \sum_{n = 0}^\infty (-1)^n
    \frac{B_0^{(2n)}}{n! (n + 1)!} \left( \frac{r}{2} \right)^{2n + 1}.
\]

В итоге, для составляющих магнитной индукции получаем:
\[
  B_{0z}(r, z) = \sum_{n = 0}^\infty (-1)^n
    \frac{B_0^{(2n)}}{(n!)^2} \left( \frac{r}{2} \right)^{2n}, \qquad
    B_{0r}(r, z) = \sum_{n = 0}^\infty (-1)^n
    \frac{B_0^{(2n + 1)}}{n! (n + 1)!} \left( \frac{r}{2} \right)^{2n + 1}.
\]

В параксиальном приближении
\[
  B_{0r} = -\frac{r}{2}B_0', \quad
    B_{0z} = B_0.
\]
