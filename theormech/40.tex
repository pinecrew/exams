\chapter{Понятие механической системы. Основные характеристики механической
системы: масса, центр масс. Внутренние силы и их свойства. Дифференциальные
уравнения движения механической системы. Теорема о движении центра масс
механической системы. Закон сохранения движения центра масс.}

Механическая система -- это совокупность материальных точек, рассматриваемых как
единое целое с позиций механического движения.

Основными характеристиками механической системы являются её \emph{масса}
\( \ds M = \sum_i m_i \) -- суммарная масса всех материальных точек,
составляющих систему, и \emph{центр масс} \( \ds \vec{r}_c~=~\frac{\sum\limits_i
m_i\vec{r}_i}{M} \).

Силы, действующие на механическую систему можно разделить на две категории:
внешние и внутренние. Внешними силами называют силы, действующие на систему со
стороны других систем, а внутренними -- силы, действующие на части системы со
стороны других частей этой же системы.

\section{Свойства внутренних сил}
\begin{enumerate}
    \item Главный вектор внутренних сил равен 0: \( \sum\vec{F}^i = 0 \), так
    как по второму закону Ньютона для \( j \)-й и \( k \)-й точек системы имеем:
    \[
        \vec{F}^i_{jk} = -\vec{F}^i_{kj}, \quad \text{следовательно,}\quad
        \sum_j\sum_k \vec{F}^i_{jk} = 0, \quad \text{или:} \quad
        \sum \vec{F}^i = 0.
    \]
    
    \item Главный момент внутренних сил равен 0: \( \sum \vec{M}^i = 0 \). Это
    следует из тех же соображений, так как \( \vec{F}^i_{jk} \) и
    \( \vec{F}^i_{kj} \) лежат на одной прямой.
\end{enumerate}

% заголовок -----------

Записывая уравнения движения для каждой точки системы, имеем
\[
    m_j \dder{\vec{r}_j}{t} = \vec{F}^e_j + \vec{F}^i_j.
\]

Суммируя их, для системы в целом получим:
\[
    \sum_j m_j \dder{\vec{r}_j}{t} = \sum_j\vec{F}^e_j + \sum_j\vec{F}^i_j,\quad
    \dder{}{t}\sum_j m_j \vec{r}_j = \sum_j \vec{F}^e_j, \quad
    \dder{}{t}(M\cdot\vec{r}_c) = \sum_j \vec{F}^e_j = \vec{R}^e.
\]

Получаем \( M\cdot\vec{a}_c = \vec{R}^e \) -- теорема о движении центра масс
механической системы.

\[
    \vec{Q} = \sum m_i\vec{v}_i = \der{}{t}\sum m_i\vec{r}_i =
    \der{}{t}(M\cdot\vec{r}_c) = M\vec{v}_c \text{ -- количество движения
    механической системы}.
\]

\( \ds \der{\vec{Q}}{t} = M\vec{a}_c = \sum\vec{F}^e \). В случае, когда
\( \sum\vec{F}^e = 0 \) имеем \( \ds \der{\vec{Q}}{t} = 0 \), следовательно,
\( \vec{Q}~=~\const \) -- закон сохранения количества движения механической
системы.

\newpage
