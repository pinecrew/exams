\question{Полные наборы физических величин}

Большую роль в квантовой механике играют наборы физических величин, обладающие
следующим свойством: эти величины измеримы одновременно, причём если они имеют
одновременно определённые значения, то уже никакая другая физическая величина
(не являющаяся их функцией) не может иметь в этом состоянии определённого
значения. Такие наборы физических величин называют полными наборами.

