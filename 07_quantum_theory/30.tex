\question{Обменное взаимодействие}
\newcommand{\bra}[1]{\ensuremath{\langle#1|}}
\newcommand{\ket}[1]{\ensuremath{|#1\rangle}}
Рассмотрим взамодействие протона и нейтрона в ядре. Считается, что они
взаимодействуют друг с другом обмениваясь \( \pi \)-мезонами. Протон испускает
\( \pi^+ \)-мезон и превращается в нейтрон, а нейтрон получает его и
превращаяется в протон. Если для простоты считать, что протон состоит из
нейтрона и \( \pi^+ \)-мезона, то можно рассматривать пару протон-нейтрон как
двухуровневую систему.
Эта система может находиться в двух ортогональных состояниях
\ket{p,n}  и \ket{n,p} и переходить между ними. Пусть амплитуда нахождения
системы в состоянии \ket{p,n} равна \( a_1(t) \), а в другом -- \( a_2(t) \).
Тогда из волнового уравнения получаем систем из двух дифференциальных уравнений:
\[
    \left\{
        \begin{array}{l}
            i\hbar\der{a_1(t)}{t} = E_0 a_1(t) - A a_2(t),\\
            i\hbar\der{a_2(t)}{t} = - A a_1(t) + E_0 a_2(t),
        \end{array}
    \right.
\]
где \( E_0 \) -- энергия каждого из состояний, \( A \) имеет смысл вероятности
перехода из одного состояния в другое в единицу времени. Нам требуются состояния
с определённой энергией -- для них \( a(t)  = ae^{-\frac{i}{\hbar}E} \).
Подставляя такой вид решения, получаем
\[
    \left\{
        \begin{array}{l}
            (E_0-E) a_1 - A a_2 = 0,\\
            - A a_1 + (E_0-E) a_2 = 0.
        \end{array}
    \right.
\]
Решая, получаем 2 уровня энергии
\[
    E_1 = E_0 - A \text{ и } E_2 = E_0 + A
\]
и соответствующие им состояния
\[
    \ket{1} = \frac{1}{\sqrt{2}}(\ket{p,n} + \ket{n,p}),\quad
    \ket{2} = \frac{1}{\sqrt{2}}(\ket{p,n} - \ket{n,p}).
\]
Таким образом, за счёт возможности перехода мезона энергия состояния уменьшается
и между частицами образуются силы притяжения. Потенциал нуклонного
взаимодействия описывается зависимостью A от r. Значение А пропорцинально
значению волновой функции мезона
\[
    A \sim \frac{e^{\frac{i}{\hbar}pr}}{r}.
\]
Но полная энергия мезона в пространстве между нуклонами близка к нулю:
\[
    E^2 = p^2c^2 + m^2c^4 = 0,
\]
следовательно, его импульс -- мнимая величина
\[
    p = imc.
\]
Такие мезоны называются "виртуальными", потому что с классической точки зрения
они не могут существовать в пространстве между нуклонами.
Подставляя это значение в выражение для вероятности перепрыгивания мезона между
нуклонами, получаем
\[
    A \sim \frac{e^{-\frac{mc}{\hbar}r}}{r}.
\]

Так же можно описать и электростатическое взаимодействие. Можно считать, что
электроны обмениваются "виртуальными" фотонами. Тогда учитывая, что фотоны не
имеют массы покоя
\[
    \phi \sim \frac{e^{0}}{r} = \frac{1}{r}.
\]
Получается результат, совпадающий с результатом классической электродинамики.

