\question{Основное уравнение квантовой механики}
Волновая функция полностью определяет состояние системы в квантовой механике.
Это означает, что задание этой функции в  некоторый момент времени позволяет
определить состояние системы во все последующие моменты времени. Математически
это выражается в том, что производная волновой функции по времени определяется
самим значением этой функции в данный момент:
\begin{equation}
    i\hbar\pder{\psi}{t} = \hat{H}\psi,
    \label{05:eq:equation}
\end{equation}
где \( \hat{H} \) -- некоторый линейный оператор, смысл которого будет определён
ниже.

Поскольку \( \int\psi^*\psi dq = \const \), то
\[
    \der{}{t}\int\psi^*\psi dq = \int\pder{\psi^*}{t}\psi dq +
    \int\psi^*\pder{\psi}{t} dq = 0.
\]

Используя (\ref{05:eq:equation}) и применив в 1 интеграле определение
транспонированного оператора, получим (с точностью до множителя \( i/\hbar \)):
\[
    \int\psi\hat{H}^*\psi^* dq - \int\psi^*\hat{H}\psi dq =
    \int\psi*\tilde{\hat{H}}\psi dq - \int\psi^*\hat{H}\psi dq =
    \int\psi^*(\tilde{\hat{H}} - \hat{H})\psi dq.
\]
Так как это соотношение должно выполняться для произвольной функции, то оператор
\( \hat{H} \) -- эрмитов.

Квазиклассическая волновая функция имеет вид
\[
    \psi = ae^{i\frac{S}{\hbar}},
\]
где \( S \) -- действие. Дифференцируя по времени, получаем
\[
    \pder{\psi}{t} = \frac{i}{\hbar}\pder{S}{t}\psi.
\]
Сравнивая с (\ref{05:eq:equation}), делаем вывод, что в квазиклассическом случае
оператор \( \hat{H} \) сводится к простому умножению на \( -\pder{S}{t} \). То
есть он имеет смысл полной энергии системы и называется гамильтонианом.

Уравнение (\ref{05:eq:equation}) называется основным уравнением квантовой
механики.
